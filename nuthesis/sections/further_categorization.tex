Since electrons and muons have different selection efficiencies and fake-rates, the background spectrum in same-sign \emu\ events is somewhat different in events where electrons or muons are the softer leg.
Hence a splitting of the channel could potentially give an improvement in the limit.
Similarly, for the three lepton channel, splitting into events with a pair of same-flavor, opposite-sign leptons (SFOS) or without (SFSS), changes the background composition somewhat, with a potential gain in sensitivity.
We calculate the limits for each channel inclusively and compare with the limit obtained when combining the split categories, see Tab.~\ref{tab:split_channel_limit}.

In the case of same-sign dileptons channels, the limit improves from 2.18 (inclusive \emu\ channel) to 2.12 after combining the exclusive channels.
The three lepton channel improves from 1.96 (inclusive channel) to 1.90 when running SFSS and SFOS separately.

\begin{table}[h!]
\centering
\begin{tabular}{lll}
Scenario   & Channel                            & Exp. Limit (median) \\\hline
$\CV=1.0$  & \emu\ (inclusive)                  & \textbf{2.18}       \\
$\Ct=-1.0$ & \emu\ (exclusive)                  & 2.59                \\
           & $\mu^\pm e^\pm$ (exclusive)        & 2.93                \\
           & Combined ($\emu+\mu^\pm e^\pm$)    & \textbf{2.12}       \\ \hline
           & \threel\ (inclusive)               & \textbf{1.96}       \\
           & \threel\ (SFSS)                    & 3.18                \\
           & \threel\ (SFOS)                    & 2.40                \\
           & Combined (SFSS, SFOS)              & \textbf{1.90}       \\ \hline
\end{tabular}
\caption{Expected limits (at 95\% C.L.) on the combined $\cPqt\PH$ production in the same-sign dilepton and three lepton channels, and for their combination, for a scenario with inverted couplings ($\CV=1.0$,$\Ct=-1.0$). Numbers are for 35.9\fbinv.}
\label{tab:split_channel_limit}
\end{table}
