\chapter{My notes}\label{ch:ananotes}
what exactly means SIDM?\\
Why no excess? is the analysis complete?\\
why not $\tau$? why not $Z_d \rightarrow \gamma\gamma$\\
how does the ABCD method works?\\
What's pdmV group?\\
Ref for the lagrangian? and explanation\\
Where did (2) and (3) come from\\
Why these range of masses? and $\epsilon$ values\\
what is NoVtx algorithm?\\
line 371 does the lepton jet decay?\\

The WIMP paradigm predicts a particle with a mass around the weak scale and with a cross section reminiscent of the weak interaction, which can naturally explain the measured dark matter abundance in the universe, assuming thermal dark matter production in the ΛCDM standard cosmological model

A lepton-jet is a group of collimated leptons in a tight cone. why include the photons?\\

\section{Start working with the code}
\section{Dan's Email}
\textbf{How are the electrons and muons identified? They are not conventional leptons, because they are not required to point to the center of the detector (they are displaced).}

{\rojo{How are lepton jets identified (what you said today seemed fine, though there were some aspects you were going to check.}}
Lepton jets refers to a clusters of lepton in a tight cone

In the ntuple we should be able to find:

\textbf{The number of electrons and number of muons found.}

\textbf{We should be able to identity which leptons belonged to lepton jets (and which jet).}

\textbf{There should be a number of lepton jets (and of the two types: electron, muon).}

\textbf{There should be events with 2 lepton jets classified as 2 electron jet events, 2 muon jet events, 1 muon jet and 1 electron jet type.  There should be an event count of how many of each type.}


