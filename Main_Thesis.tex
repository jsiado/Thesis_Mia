%% This is file `skeleton.tex',
%% generated with the docstrip utility.
%%
%% The original source files were:
%%
%% nuthesis.dtx  (with options: `skeleton')
%% For common degrees, you can use the class options:
%% phd, edd, ms, ma
%% phd is the default


\documentclass[print]{nuthesis}
%\usepackage{mathptmx} % times new roman
\usepackage{graphicx}
\graphicspath{ {nuthesis/images/} }
\usepackage{times}
\usepackage{epstopdf}
\usepackage{amsmath}
%\usepackage{program}
%\usepackage{algorithmic}
\usepackage{algorithm}
\usepackage{algpseudocode}



%%% My packages

\usepackage[english]{babel}
\usepackage[T1]{fontenc}
\usepackage{latexsym}
\usepackage{float}
\usepackage{amsthm}
\usepackage{amsfonts}
\usepackage{epsfig}
%\usepackage{subfigure} % subfiguras
\usepackage{hyperref}
\usepackage{amssymb}
%\usepackage[pdftex]{color,graphicx}
\usepackage{color} %Paquete para escribir letras en color
%\usepackage[pdftex]{hyperref}
%\usepackage[dvips]{graphicx}
%\usepackage{setspace}
%\singlespacing
%\usepackage{fancyhdr}
\usepackage[hang]{footmisc}
\usepackage[font=small,labelfont=bf]{caption}
%\usepackage{biblatex}
%\usepackage{cite}
%\usepackage[noadjust]{cite}
\usepackage{pifont}
\usepackage{multirow}

\usepackage{xspace}

\usepackage{lmodern}
\usepackage{cite}
\usepackage{lineno}
\usepackage{arydshln} %for dashed lines
\usepackage{slashed}
\usepackage{bm}
\usepackage[normalem]{ulem}
\usepackage{pdflscape}




\newcommand{\tHq}{\ensuremath{tHq}\xspace}
\newcommand{\tHW}{\ensuremath{tHW}\xspace}
\newcommand{\tH}{\ensuremath{tH}\xspace}
\newcommand{\ttH}{\ensuremath{t\bar{t}H}\xspace}
\newcommand{\ttZ}{\ensuremath{t\bar{t}Z}\xspace}
\newcommand{\ttW}{\ensuremath{t\bar{t}W}\xspace}
\newcommand{\ttV}{\ensuremath{t\bar{t}\mathrm{V}}}
\newcommand{\WW}{\ensuremath{WW}\xspace}
\newcommand{\HWW}{\ensuremath{HWW}\xspace}
\newcommand{\WZ}{\ensuremath{WZ}\xspace}
\newcommand{\ZZ}{\ensuremath{ZZ}\xspace}
\newcommand{\tZW}{\ensuremath{tZW}}
\newcommand{\tZq}{\ensuremath{tZq}}
\newcommand{\tautau}{\ensuremath{\tau\tau}\xspace}
\newcommand{\Zll}{\ensuremath{\mathrm{Z}\to\ell^+\ell^-}\xspace}
\newcommand{\Ztt}{\ensuremath{\mathrm{Z}\to\tau^+\tau^-}\xspace}
\newcommand{\tXq}{\ensuremath{tX_0q}\xspace}
\newcommand{\reliso}{\ensuremath{I_\mathrm{rel}}\xspace}
\newcommand{\sip}{\ensuremath{S_\mathrm{IP3D}}\xspace}
\newcommand{\Pgth}{\ensuremath{\Pgt_{\rm h}}\xspace}
\newcommand{\ptRatio}{\ensuremath{\pt^\text{ratio}}\xspace}
\newcommand{\ptRel}{\ensuremath{\pt^\text{rel}}\xspace}
\newcommand{\relIso}{\ensuremath{I_\text{rel}}\xspace}
\newcommand{\miniIso}{\ensuremath{I_\text{mini}}\xspace}
\newcommand{\CV}{\ensuremath{\kappa_\text{V}}\xspace}
\newcommand{\Ct}{\ensuremath{\kappa_t}\xspace}
\newcommand{\ft}{\ensuremath{f_t}\xspace}
\newcommand{\mumu}{\ensuremath{\mu^\pm\mu^\pm}\xspace}
\newcommand{\emu}{\ensuremath{\e^\pm\mu^\pm}\xspace}
\newcommand{\ee}{\ensuremath{\e^\pm \e^\pm}\xspace}
\newcommand{\threel}{\ensuremath{\ell\ell\ell}\xspace}
\newcommand{\fbinv}{fb\ensuremath{^{-1}}\xspace}
\newcommand{\ttbar}{\ensuremath{t\bar{t}}\xspace}
\newcommand{\bbbar}{\ensuremath{b\bar{b}}\xspace}
\newcommand{\pt}{\ensuremath{p_T}\xspace}
\newcommand{\Et}{\ensuremath{E_T}\xspace}
\newcommand{\bjet}{\ensuremath{b}-jet\xspace}
\newcommand{\bjets}{\ensuremath{b}-jets\xspace}
\newcommand{\ie}{i.e.\xspace}
\newcommand{\wpm}{\ensuremath{W^\pm}\xspace}
%\newcommand{\z0}{\ensuremath{z^0}\xspace}
\newcommand{\Lagr}{\mathcal{L}}
\newcommand{\beqn}{\begin{equation}}
\newcommand{\eeqn}{\end{equation}}
\newcommand{\bit}{\begin{itemize}}
\newcommand{\eit}{\end{itemize}}
\newcommand{\pp}{\ensuremath{pp}\xspace}
\newcommand{\etac}{\ensuremath{\eta}\xspace} %\etac=\eta coord
\newcommand{\phic}{\ensuremath{\phi}\xspace}
\newcommand{\dt}{\ensuremath{\mathcal{DT}}\xspace}
\newcommand{\ital}{\textit}
\newcommand{\MET}{\ensuremath{E_T^{miss}}\xspace}
\newcommand{\e}{\textrm{e}}
\newcommand{\tbf}{\textbf}
\newcommand{\rojo}{\color{red}}

\setcounter{secnumdepth}{4}

%\renewcommand*{\chapnumfont}{\normalfont\Large\bfseries\rmfamily}
%\renewcommand*{\chaptitlefont}{\normalfont\Large\scshape\bfseries\rmfamily}


%%%%%%%%%%%   chapter styles %%%%%%%%%%%%%%%%%%%%%%%%%%%%

\chapterstyle{bianchi}
\renewcommand*{\chapnamefont}{\normalfont\large\rmfamily\scshape}
\renewcommand*{\chapnumfont}{\normalfont\large}
\renewcommand*{\chaptitlefont}{\normalfont\large\rmfamily}

%\chapterstyle{crosshead}
%\chapterstyle{dowding}
%\chapterstyle{ntglike}


%%%%%%%% chapterstyle VZ21%%%%%%%%%%%%%%%%%%%%%%
%% \usepackage{calc,fourier}
%% \usepackage[T1]{fontenc}
%% \makeatletter
%% \setlength\midchapskip{7pt}
%% \makechapterstyle{VZ21}{
%%   \renewcommand\chapnamefont{\Large\scshape}
%%   \renewcommand\chapnumfont{\Large\scshape\centering}
%%   \renewcommand\chaptitlefont{\huge\bfseries\centering}
%%   \renewcommand\printchaptertitle[1]{%
%%     \setlength\tabcolsep{7pt}% used as indentation on both sides
%%     \settowidth\@tempdimc{\chaptitlefont ##1}%
%%     \setlength\@tempdimc{\textwidth-\@tempdimc-2\tabcolsep}%
%%     \chaptitlefont
%%     \ifdim\@tempdimc > 0pt\relax% one line
%%     \begin{tabular}{c}
%%         ##1\\ 
%%     \end{tabular}
%%     \else% two+ lines
%%     \begin{tabular}{%
%%         >{\chaptitlefont\arraybackslash}p{\textwidth-2\tabcolsep}}
%%        ##1\\ 
%%     \end{tabular}
%%     \fi
%%   }
%% }
%% \makeatother
%% \chapterstyle{VZ21}

%%%%%%%%%%%%%%%%%%%%%%%%%%%%%%%%%%%%%%%%%%%%%%%%%%%%%%


\addto\captionsenglish{% Replace "english" with the language you use
  \renewcommand{\contentsname}%
    {Table of Contents}%
}

\begin{document}

%\linenumbers

\renewcommand\bibname{References}
%% Start formatting the first few special pages
%% frontmatter is needed to set the page numbering correctly
\frontmatter

\title{Search for something}
\author{Joaquin Emilio Siado Casta{\~n}eda}
\adviser{Professor Daniel Claes}
\adviserAbstract{Dan Claes and Aaron Dominguez}
\major{Physics and Astronomy}
\degreemonth{July}
\degreeyear{2020}

%% For most people the following can be changed with a class
%% option. To manually set these, just uncomment the following and
%% make the needed changes.
\doctype{Dissertation}
\degree{Doctor of Philosophy}
\degreeabbreviation{Ph.D.}
%%
%% Now that we know everything we need, we can generate the title page
%% itself.
%%
\maketitle
%%
%% You have a maximum of 350 words for your abstract, which includes
%% your title, name, etc.
%%
%% Required
\begin{abstract}
 \input{nuthesis/sections/Abstract.tex}
\end{abstract}

%% Optional
%% \begin{copyrightpage}
%% \end{copyrightpage}

%% Optional
\begin{dedication}
%  \vspace{0.3\textheight}
  % \centering
 % {\huge \textit{``... Josefa.}}\\ \ \\
  %{\huge \textit{``...laca.''}}\\ {\footnotesize yo y tu.}\\ \ \\
  %{\huge \textit{``...come lady come.''}}\\ {\footnotesize pa darte.}\\
  %\vspace{4cm}
  %{\Huge $ \mathfrak{ to \ Nenas \ and \  nenita}$}
\end{dedication}

%% Optional
\begin{acknowledgments}

%\input{nuthesis/sections/Introduction.tex}

\end{acknowledgments}

%% Optional
%% \begin{grantinfo}
%% \end{grantinfo}
%% The ToC is required
%% Uncomment these if need be

%% The ToC is required
\tableofcontents*

%% Uncomment these if need be
\listoffigures
\listoftables
%%
%% ``Real'' beginning of the document. %% mainmatter is needed to set the page numbering correctly
%%   mainmatter is needed after the ToC, (LoF, and LoT) to set the %%   page numbering correctly for the main body
\mainmatter
%% Thesis goes here
%%\chapter{My Thesis}
%%\1. Introduction
%%\2. Related Work
%%\3. 3D reconstruction
%%\4. Hyperspectra data cube mining

%\hyphenation{ma-te-rials}
%%%%%%%%%%%%%%%%%%%%% Introduction %%%%%%%%%%%%%%%%%
\chapter{Everything}
\label{ch:allandall}

Seesaw mechanism
From Wikipedia, the free encyclopedia

In the theory of grand unification of particle physics, and, in particular, in theories of neutrino masses and neutrino oscillation, the seesaw mechanism is a generic model used to understand the relative sizes of observed neutrino masses, of the order of eV, compared to those of quarks and charged leptons, which are millions of times heavier.

There are several types of models, each extending the Standard Model. The simplest version, “Type 1”, extends the Standard Model by assuming two or more additional right-handed neutrino fields inert under the electroweak interaction,[a] and the existence of a very large mass scale. This allows the mass scale to be identifiable with the postulated scale of grand unification.

neutrinos are left handed

What lepton is displaced:?
why nr has two lines?
What mass ranges are we going to use?
What generator would be used?
What does it mean neutrinos are massless at tree level?
Know someone at ll meeting?

to run code %root -l my_HNL_files/Macros/RHN_Ana.C'("RHN_Mu_13TeV_10k.root")'

\section{arXiv:15041247... Theory paper, Izaguirre}
SM unanswered quetions: Origin of Neutrino masses. the identity of dark matter, and the dynamics responsible for baryon assymetry. A minimal extension of the SM with 3 gauge-singlet 'sterile' right-handed neutrinos (RHN), if they lie below the weak scale, $\nu$MSM.

RHN can be produced by any interaction involving SM neutrinos. i.e $W^{\pm} \to l^{\pm}_\alpha$. Above the bmass, the dominant production mechanism for N is via $W^{\pm} \to l^{\pm}N$ and $Z\to \nu N$

In the standard model of particle physics, the Yukawa couplings encode the interaction between the fundamental fermion fields and the Higgs field, and thus, via the Higgs mechanism, the masses of the fermion field after electroweak symmetry breaking.

DLJ: What is a Displaced LJ?
• LJ: Collimated jet-like structure containing pair(s) of muons
and/or electrons (and/or pions)
• Displaced: Produced far from primary interaction vertex of event

collimated jet-like structures containing pairs of leptons and/or light hadrons, the so-called lepton-jets (LJs).

\hyphenation{ma-te-rials}
%%%%%%%%%%%%%%%%%%%%% Introduction %%%%%%%%%%%%%%%%%
\chapter{Introduction}\label{ch:intro}

Talk about particle physics in general and the organization of the documents
\hyphenation{se-para-tion}
\hyphenation{theo-re-ti-cal}
\hyphenation{handed-ness}
\hyphenation{fo-llo-wing}
\hyphenation{ac-cor-ding}

%______________________ Theory ______________________
\chapter{The LHC Experiment and the CMS Detector}\label{ch:lhcandcms}

Small intro
\section{The LHC}
With 27 km in circunference and located in the Swiss-french border\cite{pix_tdr}
\subsection{LHCb}
\subsection{Atlas}
\subsection{ALICE}
\section{CMS}
\subsection{The Muon Detector}
\subsection{Ecal}
\subsection{HCal}
\subsection{Pixel Detector}

%______________________ INTRODUCCION ______________________
%\section{Introduction}
%\label{secc:Intro_th}




%\setcounter{chapter}{3}
\chapter{The SM and BSM Theories}\label{ch:smandbsm}

It is the most succedfull scintific theory ever written.  Proposed in the 1960s the standard model of particles physics has been successful in describing many phenomena of the particle world




{\rojo{unanswered question of sm. First goal achieve but still many other questions to solve}}






\hyphenation{ra-dia-tion}
\hyphenation{re-gu-la-ri-ty}
\hyphenation{thres-hold}
\hyphenation{ge-ne-ra-li-za-tion}
\chapter{Event generation, simulation and reconstruction}\label{ch:gensimreco}

Description of event generation and simulation
%\setcounter{chapter}{5}
\hyphenation{a-na-ly-ze}
\hyphenation{framework}
\chapter{Search for the particle}\label{ch:analyisis}

Data analysis details
\input{nuthesis/sections/Ch6-Ana-III.tex}
%\setcounter{chapter}{7}
\chapter{Module Production for the Phase I CMS Pixel Detector Upgrade}\label{ch:phase1}

As discussed in chapter \ref{ch:lhcandcms} the CMS pixel detector will \ital{suffer} from radiation damage throughout its lifetime hence the need for periodical updates. The first version of the detector was known as phase 0, it became fully operational 2010 after solving a setback during the original starting period in 2008. In 2017 the pixel detector was replaced during the so-called phase 1 upgrade, the University of Nebraska, high energy group (UNL-HEP) played a major role in assembling and testing over 500 modules, from 2013 to 2016, which then became part of the forward region of the pixel detector (FPix). The next update of this detector (phase 2) is projected to take place in 2025 when the current detector will be reaching its limits. In this chapter we describe why the phase 0 pixel detector needed an upgrade making 

the work done by the UNL-HEP group. Some of these steps will be highlighted and described in detail as they were my contributions to this production campaign. Specially the  and highlighting 





\section{The CMS Pixel Detector Phase I Upgrade}
The CMS pixel detector is composed of two sections, the barrel section (BPix) and the forward section (FPix). Each of these sections (for phase 0) was composed of three layers originally designed to record three 3D positions (tracks) of the particles emerging from the \ital{pp} collisions. As well as to provide information to reconstruct primary and secondary vertices of decaying particles. This detector performed well during the LHC run I,{\rojo{incorporate the bunch crossing?}} taking data at the design luminosity of $1 x 10^{34} cm^{-2} s^{-1}$, which was then used in many analysis including the discovery of the Higgs bosson published in 2013. But after a few years of operation the pixel detector started to degrade due to radiation damage, causing an increase of fake rates as well as loose on resolution. Moreover, for run II the LHC planned to double the luminosity with successive increment until reach its peak of $2 x 10^{35} cm^{-2} s^{-1}$. A simulation of the performance of the pixel detector under different luminosity conditions can be seen in figure \ref{fig:red_perf}

\begin{figure}[!h]
\centering
\includegraphics[width=0.9\textwidth]{ch7/reducedperformance}
%\includegraphics[width=0.9\textwidth]{pixel/reducedperformance}
\caption[Expected performance of the original pixel detector for different luminosities.]{Expected performance of the original pixel detector under different luminosity conditions: a) track-finding efficiency; b) fake rate. Conventions are the same for both plots, considering zero pileup (blue squares), average pileup of 25 (red dots), average pileup of 50 (black diamonds), and average pileup of 100 (magenta triangles).\cite{pix_tdr}}\label{fig:red_perf}
\end{figure}

This degradation prompted the need for an improved pixel detector. It was designed to have four layers in the barrel ubicated at distances of Hena2420
and 3 layers at each endcap as shown in \ref{fig:new_pix}. better 

{\rojo{This degradation prompted the need for an improved pixel detector. It was designed to have four layers in the barrel ubicated at distances of Hena2420
and 3 layers at each endcap as shown in \ref{fig:new_pix}. better }}

{\rojo{This degradation prompted the need for an improved pixel detector. It was designed to have four layers in the barrel ubicated at distances of Hena2420
and 3 layers at each endcap as shown in \ref{fig:new_pix}. better lllllllllllll l l l l lll ll l l l ll l l l ll l lllllllllllllllllllllllllllllll lll l ll l l ll ll ll lll llll llll lll llll lll lll lll lllll lll llllll lllll lll llll llll llll lllll lll ll l l l ll pasa hacer espacio tenemos que escribir vainas mas coherentes a ver si ahora si slata la pagina}}

\begin{figure}[!h]
\centering
\includegraphics[width=0.6\textwidth]{../images/ch7/fpix.pdf}
\includegraphics[width=0.39\textwidth]{../images/ch7/bpix.pdf}
\caption[Layout of the upgraded and old pixel detectors.]{Layout and comparison of the layers and disks in the upgraded (Phase I) and old (Phase 0) pixel detectors \cite{pix_tdr}.}\label{fig:new_pix}
\end{figure}

%%%%%%%%%%%%%
%%%%%%%%%%%%
%%%%%%%%%%

\section{Module Production at UNL}
The UNL module production workflow was designed to follow a pipeline-like structure as shown in figure \ref{fig:unlworkflow}. 

\begin{figure}[!h]
  \centering
  \includegraphics[width=0.7\textwidth]{../images/ch7/unl_workflow}
  \caption[UNL module assembly workflow.]{UNL module assembly workflow. Dashed lines represent occasional quality testing and reworking procedures\cite{ph1_sop}.}\label{fig:unlworkflow}
\end{figure}

This allows for different batches of modules to be going through it at different stages without stopping the workflow. Following is a short description of the tests and procedures performed during the production in the UNL silicon Lab. Special emphasis will be made in IV test, visual inspection and electrical test, the stages where the author of this work made most of the work{\rojo{improve}}. 

\subsection{Visual Inspections}
The UNL-HEP group assembly workflow started upon receiving two components: a Bare Bonded Module (BBM) and a High Density Interconnect (HDI), see figure \ref{fig:bbmyhdi}.

\begin{figure}[!h]
\centering
\includegraphics[width=0.6\textwidth]{ch7/bare_module}
\includegraphics[width=0.39\textwidth]{ch7/gato2}
\caption[Photograph of a BBM and HDI.]{Photograph of a BBM (left) and HDI (right) as received by the UNL-HEP group.}\label{fig:bbmyhdi}
\end{figure}

The first stage of the module production was to do a visual inspection on these components to ensure they were in good conditions and able to continue into the production pipeline. To get a good view of such a small components a powerful microscope with magnification of {\rojo{confirm}} and an attached camera was used. A photograph of the set up is shown in figure \ref{fig:iv_station}.

\begin{figure}[!h]
\centering
\includegraphics[width=0.7\textwidth]{ch7/iv_station}
\caption[Photograph of the visual inspection and IV test station.]{Photograph showing a BBM under the microscope during a visual inspection. This station also served as IV test stand.}\label{fig:iv_station}
\end{figure}

Figure \ref{fig:vis_insp} shows different parts of four different modules where defects are observed. Some of these defects (Fig 2(chipped) ) will cause the module to be rejected inmediately whle others (Fig 3 and 4) will still undergo an IV test. 

\begin{figure}[!h]
  \centering
  \includegraphics[width=0.4\textwidth]{ch7/vis_insp_1}
  \includegraphics[width=0.4\textwidth]{ch7/vis_insp_2}
  \includegraphics[width=0.4\textwidth]{ch7/vis_insp_3}
  \includegraphics[width=0.4\textwidth]{ch7/vis_insp_4}
  \caption[Visual inspection of a bare module.]{Photograph of the visual inspection of a BBM showing few of the things observed during a visual inspection: A good module (top left), scratches on the high voltage connection path (top right), scratch on the midle of the BBM (bottom left), and scratches on the wire bonds pad (bottom right)}\label{fig:vis_insp}
\end{figure}

{\rojo{why BBM has more scratches than the HDI}}

\subsection{IV Test}
After both BBM and HDI were visually inspected 

\begin{figure}[!h]
  \centering
  \includegraphics[width=0.7\textwidth]{ch7/sensor_probe_positions}
  \caption[bla for index.]{bla bla.}\label{fig:sensor_probe_positions}
\end{figure}

\begin{figure}[!h]
  \centering
  \includegraphics[width=0.7\textwidth]{ch7/iv_test}
  \caption[bla for index.]{bla bla.}\label{fig:vis_insp}
\end{figure}

\subsection{Gluing}
\subsection{Wirebonding}
\subsection{Encapsulation}


\begin{figure}[!h]
  \centering
  \includegraphics[width=0.7\textwidth]{../images/ch7/gantry}
  \caption[bla for index.]{bla bla.}\label{fig:gantry}
\end{figure}



\subsection{Electrical Test of a Fully assembly Module}
\begin{figure}[!h]
  \centering
  \includegraphics[width=0.7\textwidth]{../images/ch7/fully_asem_mod}
  \caption[bla for index.]{bla bla.}\label{fig:fully_asem_mod}
\end{figure}

\begin{figure}[!h]
  \centering
  \includegraphics[width=0.7\textwidth]{../images/ch7/bbm_hdi}
  \caption[Visual inspection of a bare module.]{Visual inspection of a bare module.}\label{fig:vis_insp}
\end{figure}





\begin{figure}[!h]
  \centering
  \includegraphics[width=0.7\textwidth]{../images/ch7/pix_unit_cell}
  \caption[bla for index.]{bla bla.}\label{fig:pix_unit_cell}
\end{figure}



\begin{figure}[!h]
  \centering
  \includegraphics[width=0.7\textwidth]{../images/ch7/unl_workflow2}
  \caption[bla for index.]{bla bla.}\label{fig:unl_workflow2}
\end{figure}


















%\setcounter{chapter}{6}
\chapter{Beam Test of the RD53 chip for CMS Pixel Detector Upgrade Phase 2}\label{ch:testbeam}

\section{Introduction}
\section{The RD53 Chip}
\section{Purpose of Test Beam}
\section{Test Beam Set Up}
\section{Results}
%\section{Conclusions and Future Work}
\input{nuthesis/sections/Ch9-Conclusions.tex}

%%%%%%%%%%%%%%%%%%%%%%% Future %%%%%%%%%%%%%%%%%
\chapter{FUTURE DIRECTIONS}
\label{ch:Future}



\backmatter

%% Start the correct formatting for the appendices
\appendix
%\chapter{Datasets and triggers}

\begin{table}[ht!]
  \centering 
  \begin{tabular}{l}\hline
    Dataset name  \\ \hline
    \verb|/JetHT/Run2016X-23Sep2016-vY/MINIAOD|\\          
    \verb|/HTMHT/Run2016X-23Sep2016-vY/MINIAOD|\\         
    \verb|/MET/Run2016X-23Sep2016-vY/MINIAOD|\\            
    \verb|/SingleElectron/Run2016X-23Sep2016-vY/MINIAOD|\\ 
    \verb|/SingleMuon/Run2016X-23Sep2016-vY/MINIAOD|\\   
    \verb|/SinglePhoton/Run2016X-23Sep2016-vY/MINIAOD|\\   
    \verb|/DoubleEG/Run2016X-23Sep2016-vY/MINIAOD|\\     
    \verb|/MuonEG/Run2016X-23Sep2016-vY/MINIAOD|\\     
    \verb|/DoubleMuon/Run2016X-23Sep2016-vY/MINIAOD|\\
    \verb|/Tau/Run2016B-23Sep2016-v3/MINIAOD|\\\hline
    \verb|/JetHT/Run2016H-PromptReco-v3/MINIAOD|\\        
    \verb|/HTMHT/Run2016H-PromptReco-v3/MINIAOD|\\        
    \verb|/MET/Run2016H-PromptReco-v3/MINIAOD|\\          
    \verb|/SingleElectron/Run2016H-PromptReco-v3/MINIAOD|\\
    \verb|/SingleMuon/Run2016H-PromptReco-v3/MINIAOD|\\   
    \verb|/SinglePhoton/Run2016H-PromptReco-v3/MINIAOD|\\ 
    \verb|/DoubleEG/Run2016H-PromptReco-v3/MINIAOD|\\     
    \verb|/MuonEG/Run2016H-PromptReco-v3/MINIAOD|\\       
    \verb|/DoubleMuon/Run2016H-PromptReco-v3/MINIAOD|\\\hline 
  \end{tabular}
  \caption[Full 2016 dataset.]{Full 2016 dataset used in the analysis. In the first section of the table are listed the 23Sep2016 samples; in the Run2016X-23Sep-vY label, X:B-G tag the run period while Y:1,3 tag the version of the data sample. Second section list the PromptReco version of the dataset.}\label{tab:dataset}
\end{table}

\begin{table}
  \centering \footnotesize
  \begin{tabular}{ll}\hline
    Same-sign dilepton (==2 muons)\\\hline
    \verb|HLT_Mu17_TrkIsoVVL_Mu8_TrkIsoVVL_DZ_v*|\\
    \verb|HLT_Mu17_TrkIsoVVL_TkMu8_TrkIsoVVL_DZ_v*|\\
    \verb|HLT_IsoMu22_v*|\\
    \verb|HLT_IsoTkMu22_v*|\\
    \verb|HLT_IsoMu22_eta2p1_v*| \\
    \verb|HLT_IsoTkMu22_eta2p1_v*| \\
    \verb|HLT_IsoMu24_v*| \\
    \verb|HLT_IsoTkMu24_v*|\\\hline
    Same-sign dilepton (==2 electrons)\\\hline
    \verb|HLT_Ele23_Ele12_CaloIdL_TrackIdL_IsoVL_DZ_v*|\\
    \verb|HLT_Ele27_eta2p1_WPLoose_Gsf_v*|\\
    \verb|HLT_Ele27_WPTight_Gsf_v*| \\
    \verb|HLT_Ele25_eta2p1_WPTight_Gsf_v*| \\\hline
    Same-sign dilepton (==1 muon, ==1 electron)\\\hline
    \verb|HLT_Mu23_TrkIsoVVL_Ele8_CaloIdL_TrackIdL_IsoVL_v*|\\
    \verb|HLT_Mu8_TrkIsoVVL_Ele23_CaloIdL_TrackIdL_IsoVL_v*|\\
    \verb|HLT_Mu23_TrkIsoVVL_Ele8_CaloIdL_TrackIdL_IsoVL_DZ_v*| \\
    \verb|HLT_Mu8_TrkIsoVVL_Ele23_CaloIdL_TrackIdL_IsoVL_DZ_v*| \\
    \verb|HLT_IsoMu22_v*|\\
    \verb|HLT_IsoTkMu22_v*|\\
    \verb|HLT_IsoMu22_eta2p1_v*| \\
    \verb|HLT_IsoTkMu22_eta2p1_v*| \\
    \verb|HLT_IsoMu24_v*| \\
    \verb|HLT_IsoTkMu24_v*| \\
    \verb|HLT_Ele27_WPTight_Gsf_v*| \\
    \verb|HLT_Ele25_eta2p1_WPTight_Gsf_v*| \\
    \verb|HLT_Ele27_eta2p1_WPLoose_Gsf_v*|\\\hline
    Three lepton\\\hline
    \verb|HLT_DiMu9_Ele9_CaloIdL_TrackIdL_v*|\\
    \verb|HLT_Mu8_DiEle12_CaloIdL_TrackIdL_v*|\\
    \verb|HLT_TripleMu_12_10_5_v*|\\
    \verb|HLT_Ele16_Ele12_Ele8_CaloIdL_TrackIdL_v*|\\
    \verb|HLT_Mu23_TrkIsoVVL_Ele8_CaloIdL_TrackIdL_IsoVL_v*|\\
    \verb|HLT_Mu23_TrkIsoVVL_Ele8_CaloIdL_TrackIdL_IsoVL_DZ_v*| \\
    \verb|HLT_Mu8_TrkIsoVVL_Ele23_CaloIdL_TrackIdL_IsoVL_v*|\\
    \verb|HLT_Mu8_TrkIsoVVL_Ele23_CaloIdL_TrackIdL_IsoVL_DZ_v|* \\
    \verb|HLT_Ele23_Ele12_CaloIdL_TrackIdL_IsoVL_DZ_v*|\\
    \verb|HLT_Mu17_TrkIsoVVL_Mu8_TrkIsoVVL_DZ_v*|\\
    \verb|HLT_Mu17_TrkIsoVVL_TkMu8_TrkIsoVVL_DZ_v*|\\
    \verb|HLT_IsoMu22_v*|\\
    \verb|HLT_IsoTkMu22_v*|\\
    \verb|HLT_IsoMu22_eta2p1_v*|\\
    \verb|HLT_IsoTkMu22_eta2p1_v*|\\
    \verb|HLT_IsoMu24_v*|\\
    \verb|HLT_IsoTkMu24_v*|\\
    \verb|HLT_Ele27_WPTight_Gsf_v*|\\
    \verb|HLT_Ele25_eta2p1_WPTight_Gsf_v*|\\
    \verb|HLT_Ele27_eta2p1_WPLoose_Gsf_v*|\\
    \hline
  \end{tabular}
  \caption[HLT paths]{Table of high-level triggers considered in the analysis.} \label{tab:triggers}
\end{table}

\begin{table}[!htbp]
  \centering
  \scriptsize
  \begin{tabular}{lllllll}
    &       & \multicolumn{2}{c}{\tHq} & \multicolumn{2}{c}{\tHW} & \\\hline
    \CV\ & \Ct\  & sum of    & cross         & sum of    & cross        & \\
    &       & weights   & section [pb]  & weights   & section [pb] & LHE weights       \\\hline
    1.0  & -3.0  & 35.700022 & 2.991         & 11.030445 & 0.6409       & LHEweight\_wgt[446]\\
    1.0  & -2.0  & 20.124298 & 1.706         & 5.967205  & 0.3458       & LHEweight\_wgt[447]\\
    1.0  & -1.5  & 14.043198 & 1.205         & 4.029093  & 0.2353       & LHEweight\_wgt[448]\\
    1.0  & -1.25 & 11.429338 & 0.9869        & 3.208415  & 0.1876       & LHEweight\_wgt[449]\\
    1.0  & -1.0  &           & 0.7927        &           & 0.1472       & \\
    1.0  & -0.75 & 7.054998  & 0.6212        & 1.863811  & 0.1102       & LHEweight\_wgt[450]\\
    1.0  & -0.5  & 5.294518  & 0.4723        & 1.339886  & 0.07979      & LHEweight\_wgt[451]\\
    1.0  & -0.25 & 3.818499  & 0.3505        & 0.914880  & 0.05518      & LHEweight\_wgt[452]\\
    1.0  & 0.0   & 2.627360  & 0.2482        & 0.588902  & 0.03881      & LHEweight\_wgt[453]\\
    1.0  & 0.25  & 1.719841  & 0.1694        & 0.361621  & 0.02226      & LHEweight\_wgt[454]\\
    1.0  & 0.5   & 1.097202  & 0.1133        & 0.233368  & 0.01444      & LHEweight\_wgt[455]\\
    1.0  & 0.75  & 0.759024  & 0.08059       & 0.204034  & 0.01222      & LHEweight\_wgt[456]\\
    1.0  & 1.0   & 0.705305  & 0.07096       & 0.273617  & 0.01561      & LHEweight\_wgt[457]\\
    1.0  & 1.25  & 0.936047  & 0.0839        & 0.442119  & 0.02481      & LHEweight\_wgt[458]\\
    1.0  & 1.5   & 1.451249  & 0.1199        & 0.709538  & 0.03935      & LHEweight\_wgt[459]\\
    1.0  & 2.0   & 3.335034  & 0.2602        & 1.541132  & 0.08605      & LHEweight\_wgt[460]\\
    1.0  & 3.0   & 10.516125 & 0.8210        & 4.391335  & 0.2465       & LHEweight\_wgt[461]\\\hline
    &       &           &               &           &              & \\\hline
    1.5  & -3.0  & 45.281492 & 3.845         & 13.426212 & 0.7825       & LHEweight\_wgt[462]\\
    1.5  & -2.0  & 27.606715 & 2.371         & 7.809713  & 0.4574       & LHEweight\_wgt[463]\\
    1.5  & -1.5  & 20.476088 & 1.784         & 5.594971  & 0.3290       & LHEweight\_wgt[464]\\
    1.5  & -1.25 & 17.337465 & 1.518         & 4.635978  & 0.2749       & LHEweight\_wgt[465]\\
    1.5  & -1.0  & 14.483302 & 1.287         & 3.775902  & 0.2244       & LHEweight\_wgt[466]\\
    1.5  & -0.75 & 11.913599 & 1.067         & 3.014744  & 0.1799       & LHEweight\_wgt[467]\\
    1.5  & -0.5  & 9.628357  & 0.874         & 2.352505  & 0.1410       & LHEweight\_wgt[468]\\
    1.5  & -0.25 & 7.627574  & 0.702         & 1.789184  & 0.1081       & LHEweight\_wgt[469]\\
    1.5  & 0.0   & 5.911882  & 0.5577        & 1.324946  & 0.08056      & LHEweight\_wgt[470]\\
    1.5  & 0.25  & 4.479390  & 0.4365        & 0.959295  & 0.05893      & LHEweight\_wgt[471]\\
    1.5  & 0.5   & 3.331988  & 0.3343        & 0.692727  & 0.04277      & LHEweight\_wgt[472]\\
    1.5  & 0.75  & 2.469046  & 0.2558        & 0.525078  & 0.03263      & LHEweight\_wgt[473]\\
    1.5  & 1.0   & 1.890565  & 0.2003        & 0.456347  & 0.02768      & LHEweight\_wgt[474]\\
    1.5  & 1.25  & 1.596544  & 0.1689        & 0.486534  & 0.02864      & LHEweight\_wgt[475]\\
    1.5  & 1.5   & 1.586983  & 0.1594        & 0.615638  & 0.03509      & LHEweight\_wgt[476]\\
    1.5  & 2.0   & 2.421241  & 0.2105        & 1.170602  & 0.06515      & LHEweight\_wgt[477]\\
    1.5  & 3.0   & 7.503280  & 0.5889        & 3.467546  & 0.1930       & LHEweight\_wgt[478]\\\hline
    &       &           &               &           & \\ \hline
    0.5  & -3.0  & 27.432685 & 2.260         & 8.929074  & 0.5136       & LHEweight\_wgt[479]\\
    0.5  & -2.0  & 13.956013 & 1.160         & 4.419093  & 0.2547       & LHEweight\_wgt[480]\\
    0.5  & -1.5  & 8.924438  & 0.7478        & 2.757611  & 0.1591       & LHEweight\_wgt[481]\\
    0.5  & -1.25 & 6.835341  & 0.5726        & 2.075247  & 0.1204       & LHEweight\_wgt[482]\\
    0.5  & -1.0  & 5.030704  & 0.4273        & 1.491801  & 0.08696      & LHEweight\_wgt[483]\\
    0.5  & -0.75 & 3.510528  & 0.2999        & 1.007273  & 0.05885      & LHEweight\_wgt[484]\\
    0.5  & -0.5  & 2.274811  & 0.1982        & 0.621663  & 0.03658      & LHEweight\_wgt[485]\\
    0.5  & -0.25 & 1.323555  & 0.1189        & 0.334972  & 0.01996      & LHEweight\_wgt[486]\\
    0.5  & 0.0   & 0.656969  & 0.06223       & 0.147253  & 0.008986     & LHEweight\_wgt[487]\\
    0.5  & 0.25  & 0.274423  & 0.02830       & 0.058342  & 0.003608     & LHEweight\_wgt[488]\\
    0.5  & 0.5   & 0.176548  & 0.01778       & 0.068404  & 0.003902     & LHEweight\_wgt[489]\\
    0.5  & 0.75  & 0.363132  & 0.03008       & 0.177385  & 0.009854     & LHEweight\_wgt[490]\\
    0.5  & 1.0   & 0.834177  & 0.06550       & 0.385283  & 0.02145      & LHEweight\_wgt[491]\\
    0.5  & 1.25  & 1.589682  & 0.1241        & 0.692099  & 0.03848      & LHEweight\_wgt[492]\\
    0.5  & 1.5   & 2.629647  & 0.2047        & 1.097834  & 0.06136      & LHEweight\_wgt[493]\\
    0.5  & 2.0   & 5.562958  & 0.4358        & 2.206057  & 0.1246       & LHEweight\_wgt[494]\\
    0.5  & 3.0   & 14.843102 & 1.177         & 5.609519  & 0.3172       & LHEweight\_wgt[495]\\ \hline
  \end{tabular}
  \caption[\CV\ and \Ct\ combinations.]{\CV\ and \Ct\ combinations generated for the two signal samples and their NLO cross sections. The \tHq\ cross section is multiplied by the branching fraction of the enforced leptonic decay of the top quark (0.324) \cite{THQProdTwiki}.}\label{tab:reweight}
\end{table}

\begin{table}
  \footnotesize
  \centering \scriptsize
  \begin{tabular}{ll}
    Sample                                                                          & $\sigma$ [pb] \\\hline
    \verb|TTWJetsToLNu_TuneCUETP8M1_13TeV-amcatnloFXFX-madspin-pythia8|             & 0.2043 \\
    \verb|TTZToLLNuNu_M-10_TuneCUETP8M1_13TeV-amcatnlo-pythia8|                     & 0.2529 \\
    \verb|/store/cmst3/group/susy/gpetrucc/13TeV/u/TTLL_m1to10_LO_NoMS_for76X/|     & 0.0283 \\
    \verb|WGToLNuG_TuneCUETP8M1_13TeV-madgraphMLM-pythia8|                          & 585.8 \\
    \verb|ZGTo2LG_TuneCUETP8M1_13TeV-amcatnloFXFX-pythia8|                          & 131.3 \\
    \verb|TGJets_TuneCUETP8M1_13TeV_amcatnlo_madspin_pythia8|                       & 2.967 \\
    \verb|TGJets_TuneCUETP8M1_13TeV_amcatnlo_madspin_pythia8|                       & 2.967 \\
    \verb|TTGJets_TuneCUETP8M1_13TeV-amcatnloFXFX-madspin-pythia8|                  & 3.697 \\
    \verb|WpWpJJ_EWK-QCD_TuneCUETP8M1_13TeV-madgraph-pythia8|                       & 0.03711 \\
    \verb|ZZZ_TuneCUETP8M1_13TeV-amcatnlo-pythia8|                                  & 0.01398 \\
    \verb|WWZ_TuneCUETP8M1_13TeV-amcatnlo-pythia8|                                  & 0.1651 \\
    \verb|WZZ_TuneCUETP8M1_13TeV-amcatnlo-pythia8|                                  & 0.05565 \\
    \verb|WW_DoubleScattering_13TeV-pythia8|                                        & 1.64 \\
    \verb|tZq_ll_4f_13TeV-amcatnlo-pythia8_TuneCUETP8M1|                            & 0.0758 \\
    \verb|ST_tWll_5f_LO_13TeV-MadGraph-pythia8|                                     & 0.01123 \\
    \verb|TTTT_TuneCUETP8M1_13TeV-amcatnlo-pythia8|                                 & 0.009103 \\
    \verb|WZTo3LNu_TuneCUETP8M1_13TeV-powheg-pythia8|                               & 4.4296 \\
    \verb|ZZTo4L_13TeV_powheg_pythia8|                                              & 1.256 \\ \hline
    \verb|TTJets_SingleLeptFromTbar_TuneCUETP8M1_13TeV-madgraphMLM-pythia8|         & 182.1754 \quad *  \\
    \verb|TTJets_SingleLeptFromT_TuneCUETP8M1_13TeV-madgraphMLM-pythia8|            & 182.1754 \quad *  \\
    \verb|TTJets_DiLept_TuneCUETP8M1_13TeV-madgraphMLM-pythia8|                     & 87.3  \quad *\\
    \verb|DYJetsToLL_M-10to50_TuneCUETP8M1_13TeV-amcatnloFXFX-pythia8|              & 18610 \\
    \verb|DYJetsToLL_M-50_TuneCUETP8M1_13TeV-madgraphMLM-pythia8|                   & 6024 \\
    \verb|WJetsToLNu_TuneCUETP8M1_13TeV-amcatnloFXFX-pythia8|                       & 61526.7 \\
    \verb|ST_tW_top_5f_inclusiveDecays_13TeV-powheg-pythia8_TuneCUETP8M1|           & 35.6 \\
    \verb|ST_tW_antitop_5f_inclusiveDecays_13TeV-powheg-pythia8_TuneCUETP8M1|       & 35.6 \\
    \verb|ST_t-channel_4f_leptonDecays_13TeV-amcatnlo-pythia8_TuneCUETP8M1|         & 70.3144\\
    \verb|ST_t-channel_antitop_4f_leptonDecays_13TeV-powheg-pythia8_TuneCUETP8M1|   & 26.2278\\
    \verb|ST_s-channel_4f_leptonDecays_13TeV-amcatnlo-pythia8_TuneCUETP8M1|         & 3.68064 \\
    \verb|WWTo2L2Nu_13TeV-powheg|                                                   & 10.481 \\\hline
    \verb|ttWJets_13TeV_madgraphMLM|                                                & 0.6105 \\
    \verb|ttZJets_13TeV_madgraphMLM|                                                & 0.5297/0.692 \\\hline
    
  \end{tabular}
  \caption[List of background samples used in this analysis (CMSSW 80X).]{List of background samples used in this analysis (CMSSW 80X). The first section of the table lists the samples used in simulation to extract the final yields and shapes; the second section lists the samples of the processes for which the yields are estimated from data. The MC simulation is used to design the data driven methods and in the derivation of the associated systematic uncertainties. The third section lists the leading order \ttW and \ttZ samples, which in addition to the ones market with a *, where used in the BDT training.} \label{tab:bgsamples}
\end{table}

\begin{table}[h!]
  \centering
  \footnotesize
  \begin{tabular}{rr|ccc|cc}
        $f_t$  & \Ct/\CV\ & Bg-only exp. & SM exp. & Obs.\ lim. & Best fit $r$ [pb] & Best fit $\sigma$ \\ \hline
        -0.973 & -6.000 & $0.328~_{-0.090}^{+0.136}$ & $0.507~_{-0.158}^{+0.206}$ & 0.603 & $0.013~_{-0.007}^{+0.007}$ & $0.305~_{-0.169}^{+0.155}$  \\
        -0.941 & -4.000 & $0.335~_{-0.098}^{+0.137}$ & $0.509~_{-0.166}^{+0.215}$ & 0.627 & $0.036~_{-0.020}^{+0.018}$ & $0.322~_{-0.174}^{+0.157}$  \\
        -0.900 & -3.000 & $0.335~_{-0.096}^{+0.138}$ & $0.510~_{-0.172}^{+0.215}$ & 0.639 & $0.075~_{-0.039}^{+0.036}$ & $0.334~_{-0.173}^{+0.160}$  \\
        -0.862 & -2.500 & $0.334~_{-0.097}^{+0.139}$ & $0.505~_{-0.173}^{+0.217}$ & 0.649 & $0.119~_{-0.061}^{+0.056}$ & $0.341~_{-0.174}^{+0.160}$  \\
        -0.800 & -2.000 & $0.330~_{-0.095}^{+0.141}$ & $0.500~_{-0.176}^{+0.212}$ & 0.656 & $0.202~_{-0.103}^{+0.097}$ & $0.345~_{-0.176}^{+0.165}$  \\
        -0.692 & -1.500 & $0.325~_{-0.095}^{+0.139}$ & $0.485~_{-0.172}^{+0.209}$ & 0.660 & $0.369~_{-0.191}^{+0.178}$ & $0.340~_{-0.176}^{+0.164}$  \\
        -0.640 & -1.333 & $0.325~_{-0.097}^{+0.139}$ & $0.482~_{-0.173}^{+0.210}$ & 0.659 & $0.456~_{-0.238}^{+0.231}$ & $0.334~_{-0.174}^{+0.169}$  \\
        -0.610 & -1.250 & $0.321~_{-0.095}^{+0.140}$ & $0.474~_{-0.169}^{+0.210}$ & 0.653 & $0.505~_{-0.272}^{+0.252}$ & $0.328~_{-0.177}^{+0.164}$  \\
   \tbf{-0.500} & \tbf{-1.000} & $\mathbf{0.315~_{-0.093}^{+0.142}}$ & $\mathbf{0.450~_{-0.160}^{+0.213}}$ & \tbf{0.638} & $\mathbf{0.685~_{-0.396}^{+0.395}}$ & $\mathbf{0.304~_{-0.176}^{+0.175}}$  \\ 
        -0.410 & -0.833 & $0.312~_{-0.095}^{+0.138}$ & $0.424~_{-0.147}^{+0.210}$ & 0.615 & $0.819~_{-0.526}^{+0.498}$ & $0.276~_{-0.177}^{+0.168}$ \\
        -0.360 & -0.750 & $0.307~_{-0.093}^{+0.138}$ & $0.409~_{-0.136}^{+0.200}$ & 0.593 & $0.874~_{-0.601}^{+0.581}$ & $0.256~_{-0.176}^{+0.170}$ \\
        -0.308 & -0.667 & $0.301~_{-0.092}^{+0.138}$ & $0.384~_{-0.124}^{+0.198}$ & 0.566 & $0.915~_{-0.689}^{+0.655}$ & $0.231~_{-0.174}^{+0.165}$ \\
        -0.200 & -0.500 & $0.292~_{-0.090}^{+0.136}$ & $0.345~_{-0.109}^{+0.181}$ & 0.497 & $0.895~_{-0.871}^{+0.879}$ & $0.166~_{-0.162}^{+0.163}$ \\
        -0.100 & -0.333 & $0.278~_{-0.086}^{+0.132}$ & $0.303~_{-0.092}^{+0.156}$ & 0.409 & $0.679~_{-0.679}^{+1.159}$ & $0.092~_{-0.092}^{+0.157}$ \\
        -0.059 & -0.250 & $0.268~_{-0.083}^{+0.129}$ & $0.283~_{-0.085}^{+0.152}$ & 0.365 & $0.515~_{-0.515}^{+1.285}$ & $0.059~_{-0.059}^{+0.148}$ \\
        -0.027 & -0.167 & $0.260~_{-0.081}^{+0.125}$ & $0.266~_{-0.077}^{+0.135}$ & 0.328 & $0.297~_{-0.297}^{+1.434}$ & $0.029~_{-0.029}^{+0.142}$ \\
         0.000 &  0.000 & $0.254~_{-0.079}^{+0.123}$ & $0.252~_{-0.073}^{+0.123}$ & 0.294 & $0.002~_{-0.002}^{+1.776}$ & $0.000~_{-0.000}^{+0.132}$ \\
         0.027 &  0.167 & $0.275~_{-0.086}^{+0.132}$ & $0.284~_{-0.084}^{+0.148}$ & 0.357 & $0.650~_{-0.650}^{+2.514}$ & $0.040~_{-0.040}^{+0.154}$ \\
         0.059 &  0.250 & $0.297~_{-0.093}^{+0.141}$ & $0.329~_{-0.099}^{+0.171}$ & 0.458 & $2.015~_{-2.015}^{+3.098}$ & $0.119~_{-0.119}^{+0.183}$ \\
         0.100 &  0.333 & $0.322~_{-0.099}^{+0.148}$ & $0.405~_{-0.135}^{+0.220}$ & 0.611 & $4.147~_{-3.103}^{+2.802}$ & $0.246~_{-0.184}^{+0.166}$ \\
         0.200 &  0.500 & $0.324~_{-0.096}^{+0.141}$ & $0.505~_{-0.181}^{+0.212}$ & 0.730 & $5.982~_{-2.559}^{+2.174}$ & $0.413~_{-0.177}^{+0.150}$ \\
         0.308 &  0.667 & $0.281~_{-0.082}^{+0.122}$ & $0.462~_{-0.159}^{+0.172}$ & 0.651 & $4.186~_{-1.574}^{+1.492}$ & $0.382~_{-0.144}^{+0.136}$ \\
         0.360 &  0.750 & $0.268~_{-0.079}^{+0.116}$ & $0.442~_{-0.154}^{+0.160}$ & 0.620 & $3.392~_{-1.253}^{+1.214}$ & $0.364~_{-0.135}^{+0.130}$ \\
         0.410 &  0.833 & $0.258~_{-0.075}^{+0.112}$ & $0.427~_{-0.147}^{+0.162}$ & 0.599 & $2.754~_{-1.022}^{+0.999}$ & $0.351~_{-0.130}^{+0.127}$ \\
    \tbf{0.500} & \tbf{ 1.000} & $\mathbf{0.244~_{-0.072}^{+0.105}}$ & $\mathbf{0.401~_{-0.137}^{+0.154}}$ & \tbf{0.562} & $\mathbf{1.821~_{-0.671}^{+0.657}}$ & $\mathbf{0.328~_{-0.121}^{+0.118}}$\\
         0.610 &  1.250 & $0.240~_{-0.070}^{+0.104}$ & $0.394~_{-0.133}^{+0.154}$ & 0.545 & $1.072~_{-0.403}^{+0.399}$ & $0.315~_{-0.119}^{+0.118}$ \\
         0.640 &  1.333 & $0.242~_{-0.071}^{+0.105}$ & $0.398~_{-0.136}^{+0.156}$ & 0.547 & $0.921~_{-0.352}^{+0.354}$ & $0.316~_{-0.121}^{+0.122}$ \\
         0.692 &  1.500 & $0.244~_{-0.071}^{+0.106}$ & $0.401~_{-0.136}^{+0.159}$ & 0.543 & $0.678~_{-0.261}^{+0.262}$ & $0.312~_{-0.120}^{+0.120}$ \\
         0.800 &  2.000 & $0.256~_{-0.075}^{+0.109}$ & $0.416~_{-0.138}^{+0.169}$ & 0.552 & $0.317~_{-0.129}^{+0.123}$ & $0.311~_{-0.127}^{+0.121}$ \\
         0.862 &  2.500 & $0.268~_{-0.078}^{+0.114}$ & $0.433~_{-0.142}^{+0.169}$ & 0.558 & $0.170~_{-0.072}^{+0.070}$ & $0.310~_{-0.130}^{+0.127}$ \\
         0.900 &  3.000 & $0.276~_{-0.080}^{+0.118}$ & $0.442~_{-0.144}^{+0.177}$ & 0.563 & $0.102~_{-0.044}^{+0.042}$ & $0.308~_{-0.134}^{+0.128}$ \\
         0.941 &  4.000 & $0.290~_{-0.084}^{+0.122}$ & $0.459~_{-0.149}^{+0.184}$ & 0.566 & $0.046~_{-0.021}^{+0.020}$ & $0.304~_{-0.140}^{+0.134}$ \\
         0.973 &  6.000 & $0.306~_{-0.081}^{+0.122}$ & $0.474~_{-0.150}^{+0.192}$ & 0.571 & $0.016~_{-0.008}^{+0.007}$ & $0.300~_{-0.150}^{+0.131}$ \\
    \hline
  \end{tabular}
  \caption[Expected and observed upper limits.]{Expected (for background only, and for a SM-like Higgs signal) and observed 95\% C.L. upper limits (in pb), and best fit signal strength $r$ and corresponding best fit cross section for the combined $\tH+\ttH$ cross section times modified branching ratio for the combination of all three channels, for different values of $\Ct/\CV$ or the equivalent $\ft$ numbers.}
  \label{tab:xslimits}
\end{table}

%\chapter{Aditional plots}

\section{Pre-selection kinematic variables} \label{app:presel_plots}

Figures~\ref{fig:input_vars_3l_xsec}, \ref{fig:input_vars_2lss_xsec_mumu} and~\ref{fig:input_vars_2lss_xsec_emu} show the distributions of some relevant kinematic variables, normalized to the cross section of the respective processes and to the integrated luminosity.
\newpage
\begin{figure} [!h]
  \centering
  \includegraphics[width=0.26\textwidth]{3lsignal/Lep3Pt.pdf}
  \includegraphics[width=0.26\textwidth]{3lsignal/dEtaFwdJetBJet_40.pdf}
  \includegraphics[width=0.26\textwidth]{3lsignal/dEtaFwdJet2BJet_40.pdf}\\
  \includegraphics[width=0.26\textwidth]{3lsignal/dEtaFwdJetClosestLep_40.pdf} 
  \includegraphics[width=0.26\textwidth]{3lsignal/dPhiHighestPtSSPair.pdf}
  \includegraphics[width=0.26\textwidth]{3lsignal/maxEtaJet25_40.pdf}\\
  \includegraphics[width=0.26\textwidth]{3lsignal/minDRll.pdf}
  \includegraphics[width=0.26\textwidth]{3lsignal/nJet25.pdf} 
  \includegraphics[width=0.26\textwidth]{3lsignal/nJetEta1_40.pdf}\\
  \includegraphics[width=0.26\textwidth]{3lsignal/totCharge.pdf}
  \caption[Input variables to the BDT, $3l$ channel.]{Distributions of input variables to the BDT for signal discrimination, three lepton channel, normalized to their cross section and to 35.9 \fbinv.}
  \label{fig:input_vars_3l_xsec}
\end{figure}

\begin{figure} [!h]
  \centering
  \includegraphics[width=0.26\textwidth]{signalregion_2lss/mumu/Lep2Pt.pdf}
  \includegraphics[width=0.26\textwidth]{signalregion_2lss/mumu/dEtaFwdJetBJet_40.pdf}
  \includegraphics[width=0.26\textwidth]{signalregion_2lss/mumu/dEtaFwdJet2BJet_40.pdf}\\
  \includegraphics[width=0.26\textwidth]{signalregion_2lss/mumu/dEtaFwdJetClosestLep_40.pdf}
  \includegraphics[width=0.26\textwidth]{signalregion_2lss/mumu/dPhiHighestPtSSPair.pdf}
  \includegraphics[width=0.26\textwidth]{signalregion_2lss/mumu/maxEtaJet25_40.pdf}\\
  \includegraphics[width=0.26\textwidth]{signalregion_2lss/mumu/minDRll.pdf}
  \includegraphics[width=0.26\textwidth]{signalregion_2lss/mumu/nJet25.pdf} 
  \includegraphics[width=0.26\textwidth]{signalregion_2lss/mumu/nJetEta1_40.pdf}\\
  \includegraphics[width=0.26\textwidth]{signalregion_2lss/mumu/totCharge.pdf}
  \caption[Input variables to the BDT, $2lss - \mumu$ channel]{Distributions of input variables to the BDT for signal discrimination, in \mumu\ channel, normalized to their cross section and to 35.9 \fbinv.}
  \label{fig:input_vars_2lss_xsec_mumu}
\end{figure}

\begin{figure} [!h]
  \centering
  \includegraphics[width=0.26\textwidth]{signalregion_2lss/emu/Lep2Pt.pdf}
  \includegraphics[width=0.26\textwidth]{signalregion_2lss/emu/dEtaFwdJetBJet_40.pdf}
  \includegraphics[width=0.26\textwidth]{signalregion_2lss/emu/dEtaFwdJet2BJet_40.pdf}\\
  \includegraphics[width=0.26\textwidth]{signalregion_2lss/emu/dEtaFwdJetClosestLep_40.pdf} 
  \includegraphics[width=0.26\textwidth]{signalregion_2lss/emu/dPhiHighestPtSSPair.pdf}
  \includegraphics[width=0.26\textwidth]{signalregion_2lss/emu/maxEtaJet25_40.pdf}\\
  \includegraphics[width=0.26\textwidth]{signalregion_2lss/emu/minDRll.pdf}
  \includegraphics[width=0.26\textwidth]{signalregion_2lss/emu/nJet25.pdf} 
  \includegraphics[width=0.26\textwidth]{signalregion_2lss/emu/nJetEta1_40.pdf}\\
  \includegraphics[width=0.26\textwidth]{signalregion_2lss/emu/totCharge.pdf}
  \caption[Input variables to the BDT, $2lss-\emu$ channel]{Distributions of input variables to the BDT for signal discrimination, in $\emu$ channel, normalized to their cross section and to 35.9 \fbinv.}
  \label{fig:input_vars_2lss_xsec_emu}
\end{figure}

%% \begin{figure} [!h]
%%   \centering
%%   \includegraphics[width=0.22\textwidth]{signalregion_2lss/ee/Lep2Pt.pdf}
%%   \includegraphics[width=0.22\textwidth]{signalregion_2lss/ee/dEtaFwdJetBJet_40.pdf}
%%   \includegraphics[width=0.22\textwidth]{signalregion_2lss/ee/dEtaFwdJet2BJet_40.pdf}
%%   \includegraphics[width=0.22\textwidth]{signalregion_2lss/ee/dEtaFwdJetClosestLep_40.pdf} \\
%%   \includegraphics[width=0.22\textwidth]{signalregion_2lss/ee/dPhiHighestPtSSPair.pdf}
%%   \includegraphics[width=0.22\textwidth]{signalregion_2lss/ee/maxEtaJet25_40.pdf}
%%   \includegraphics[width=0.22\textwidth]{signalregion_2lss/ee/minDRll.pdf}
%%   \includegraphics[width=0.22\textwidth]{signalregion_2lss/ee/nJet25.pdf} \\
%%   \includegraphics[width=0.22\textwidth]{signalregion_2lss/ee/nJetEta1_40.pdf}
%%   \includegraphics[width=0.22\textwidth]{signalregion_2lss/ee/totCharge.pdf}
%%   \caption{Distributions of input variables to the BDT for signal discrimination, in $\ee$ channel, normalized to their cross section and to 35.9\fbinv.}
%%   \label{fig:input_vars_2lss_xsec_ee}
%% \end{figure}

\section{BDTG input variables for $2lss$ channel }

\begin{figure} [!h]
  \centering
  \includegraphics[width=0.32\textwidth]{Lep2Pt_mumu.pdf}
  \includegraphics[width=0.32\textwidth]{dEtaFwdJetBJet_mumu.pdf}
  \includegraphics[width=0.32\textwidth]{dEtaFwdJet2BJet_mumu.pdf}\\
  \includegraphics[width=0.32\textwidth]{dEtaFwdJetClosestLep_mumu.pdf}
  \includegraphics[width=0.32\textwidth]{dPhiHighestPtSSPair_mumu.pdf}
  \includegraphics[width=0.32\textwidth]{maxEtaJet25_mumu.pdf}\\
  \includegraphics[width=0.32\textwidth]{minDRll_mumu.pdf}
  \includegraphics[width=0.32\textwidth]{nJet25_mumu.pdf}
  \includegraphics[width=0.32\textwidth]{nJetEta1_mumu.pdf}\\
  \includegraphics[width=0.32\textwidth]{totCharge_mumu.pdf}
  \caption[Input variables to the BDT, $2lss$ channel]{Distributions of input variables to the BDT for signal discrimination, normalized to the equal area, for the $2lss$ channel.}
  \label{fig:input_vars_2lss}
\end{figure}  

\newpage

\section{Input variables distributions from BDTG classifiers}

\begin{figure} [!h]
  \centering
  \includegraphics[width=\textwidth]{6var_tt.pdf}
  \includegraphics[width=0.66\textwidth]{4var_tt.pdf}
  \caption[BDT input variables. Discrimination against \ttbar\ in $2lss$ channel.]{BDT input variables as seen by BDTG classifier for the $2lss$ channel, \tHq signal (blue) discriminated against \ttbar\ background (red).}
  \label{mva_input_2lss_tt}
\end{figure}

\begin{figure} [!h]
  \centering
  \includegraphics[width=\textwidth]{6var_ttv.pdf}
  \includegraphics[width=0.66\textwidth]{4var_ttv.pdf}
  \caption[BDT input variables. Discrimination against \ttV\ in $2lss$ channel.]{BDT input variables as seen by BDTG classifier for the $2lss$ channel, \tHq signal(blue) discriminated against \ttV\ background (red).}
  \label{mva_input_2lss_ttv}
\end{figure}

\begin{figure} [!h]
  \centering
  \includegraphics[width=\textwidth]{mva_input1_tt.pdf}
  \includegraphics[width=\textwidth]{mva_input2_tt.pdf}
  \caption[BDT input variables. Discrimination against \ttbar in $3l$ channel.]{BDT input variables as seen by BDTG classifier for the $3l$ channel, \tHq signal (blue) discriminated against \ttbar\ background (red).}
  \label{mva_input_tt}
\end{figure}

\begin{figure} [!h]
  \centering
  \includegraphics[width=\textwidth]{mva_input1_ttv.pdf}
  \includegraphics[width=\textwidth]{mva_input2_ttv.pdf}
  \caption[BDT input variables. Discrimination against \ttV\ in $3l$ channel.]{BDT input variables as seen by BDTG classifier for the $3l$ channel, \tHq signal (blue) discriminated against \ttV\ background (red).}
\label{mva_input_ttv}
\end{figure}

\clearpage
\section{Pulls and impacts}\label{pulls_impacts_add}

\begin{figure} [!th]
  \centering  
  \includegraphics[width=0.75\textwidth,height=0.42\textheight]{limits/impacts/impacts2.pdf}\\
  \includegraphics[width=0.75\textwidth,height=0.42\textheight]{limits/impacts/sm/impacts2.pdf}
  \caption[Additional post-fit pulls and impacts.]{Post-fit pulls and impacts of the next 20 nuisance parameters with largest impacts for the fit on the observed data, for the ITC (top) and SM (bottom) hypotheses. Continuation of pulls and impacts shown in Figure \ref{fig:impacts}}
  \label{fig:impacts2}
\end{figure}

  \begin{figure} [!h]
    \centering
    \includegraphics[width=0.75\textwidth]{limits/impacts/asimov/impacts2.pdf}\\
    \caption[Additional post-fit pulls an impacts for a fit to the Asimov dataset.]{Post-fit pulls and impacts of the next 20 nuisance parameters with largest impacts for a fit to the Asimov dataset with fixed signal strength, for the $\Ct/\CV=-1.0$ hypothesis. Continuation of pulls and impacts shown in Figure \ref{fig:impacts_asimov2}}
    \label{fig:impacts_asimov2}
  \end{figure}

%\chapter{Binning and selection optimization}\label{app:ad_binning}

\section{Binning and selection optimization}\label{app:binopt}

The effect of the choice of pre-selection cuts and the number of bins of the 1D histogram on the cross section limit is evaluated by varying the most important cuts and re-calculating the limit in each case. In this analysis, the optimization was performed in the $3l$ channel, by evaluating the upper limits on the \tHq+\ \tHW\ expected signal strength only (without \ttH component), always evaluated at $\Ct=-1.0$, $\CV=1.0$.

Table~\ref{cut_limit} shows several variations explored, compared with a baseline; the baseline is similar to the selection reported in Table~\ref{tab:cuts} but only a loose CSV jet and a Z veto of $\pm10$ GeV are required. 

\begin{table}[h!]
\centering
\begin{tabular}{lll}
Selection                         & Variation                & Expected limit \\ \hline
Baseline                          &                          & $<2.93$\\
Loose CSV tags                    & $\geq 1 \to \geq 2$      & $<3.81$\\
Medium CSV tags                   & $\geq 0 \to \geq 1$      & $<2.76$\\
Light forward jet $\eta$          & $\geq 0 \to \geq 1$      & $<2.94$\\
Light forward jet $\eta$          & $\geq 0 \to \geq 1.5$    & $<3.00$\\
MET>30 GeV                        &                          & $<2.91$\\
Z veto ($|m_{\ell\ell}-m_Z|$)     & $>10$GeV $\to >15$ GeV   & $<2.79$\\
One medium CSV + 15 GeV\ Z veto   & combined                 & $<2.62$\\\hline
\end{tabular}
\caption[Selection cuts optimization.]{Signal strength limit variation as a function of tighter cuts. The baseline selection corresponds to a looser selection compared to the one reported in Table ~\ref{tab:cuts} where only a CSV-loose \bjet is required, and the Z veto is loosened to $\pm10$ GeV. The optimal selection determined here corresponds to the baseline plus the two variations in the last row.}
\label{cut_limit}
\end{table}

The optimal limit is found when requiring a slightly tighter selection with respect to the baseline. The optimal selection is reported in Table~\ref{tab:cuts}.

The signal strength limit also depends on the chosen binning in the 2D plane as the S/B ratio varies across the plane, hence, several sizes and binning combinations were tested in order to improve the limit. Figure~\ref{bins} shows some of the binning combinations tested; in the default combination all the bins have the same size, while the best limit was found for a set of 10 bins. The bin borders and the resulting limits are shown in Table ~\ref{bin_limits}.

\begin{figure} [!h]
 \centering
 \includegraphics[width=\textwidth]{bin_scheme.pdf} 
\caption{Binning combination scheme.}
\label{bins}
\end{figure}

\begin{table}[h!]
\centering
\begin{tabular}{llllllll}\hline
Number of bins  & \multicolumn{6}{c}{Bin borders}  & Expected limit \\%\hline 
                &$x_1$&$x_2$&$x_3$&$y_1$&$y_2$&$y_3$&\\\hline           
16 (default)    &-0.5 & 0.0 & 0.5 &-0.5 & 0.0 & 0.5 & $<2.91$\\
16              &-0.5 & 0.3 & 0.7 &-0.5 & 0.3 & 0.7 & $<2.83$\\
10              &-0.5 & 0.0 & 0.5 &-0.5 & 0.0 & 0.5 & $<2.93$\\
10              &-0.5 & 0.0 & 0.7 &-0.5 & 0.0 & 0.7 & $<2.86$\\
10              &-0.5 & 0.0 & 0.7 &-0.5 & 0.0 & 0.5 & $<2.84$\\
10              &-0.5 & 0.0 & 0.5 &-0.5 & 0.0 & 0.7 & $<2.87$\\
\textbf{10}     &\textbf{-0.5} &\textbf{0.4} &\textbf{0.7} &\textbf{-0.5} &\textbf{0.4} &\textbf{0.7} &$\mathbf{<2.81}$\\\hline
\end{tabular}
\caption[Limit variation as a function of bin size, $3l$ channel.]{Limit variation as a function of bin size. The final bin borders used in the $3l$ channel are indicated in bold.}
\label{bin_limits}
\end{table}

Combining the optimization of binning and using the tighter pre-selection cuts, the expected limit in the $3l$ channel alone reaches \textbf{r<2.59}.

A similar binning optimization was made for $2lss$ channel, including other binning combinations. First, the $3l$ channel binning was used to estimate the expected limit, then, bin borders were varied to obtain the best possible expected limit. The bin borders and the resulting signal strength limits for the same-sign dimuon channel are shown in Table~\ref{bin_limits_2lss}.

\begin{table}[h!]
\centering
\begin{tabular}{llllllll}\hline
Number of bins  & \multicolumn{6}{c}{Bin borders}  & Expected limit \\
                &$x_1$&$x_2$&$x_3$&$y_1$&$y_2$&$y_3$&\\\hline
16              &-0.5 & 0.4 & 0.7 &-0.5 & 0.4 & 0.7 & $<1.72$\\
12              &-0.5 & 0.4 & 0.7 &-0.5 & 0.4 & 0.7 & $<1.72$\\
12              &-0.3 & 0.4 & 0.7 &-0.5 & 0.4 & 0.7 & $<1.71$\\
12              &-0.3 & 0.3 & 0.7 &-0.5 & 0.4 & 0.7 & $<1.71$\\
12              &-0.3 & 0.3 & 0.7 &-0.4 & 0.4 & 0.7 & $<1.70$\\
12              &-0.3 & 0.3 & 0.7 &-0.3 & 0.4 & 0.7 & $<1.70$\\
12              &-0.3 & 0.3 & 0.7 &-0.3 & 0.2 & 0.7 & $<1.68$\\
12              &-0.3 & 0.3 & 0.7 &-0.3 & 0.1 & 0.7 & $<1.70$\\
12              &-0.3 & 0.3 & 0.7 &-0.3 & 0.2 & 0.6 & $<1.70$\\
10              &-0.5 & 0.4 & 0.7 &-0.5 & 0.4 & 0.7 & $<1.75$\\
\textbf{10}     &\textbf{-0.3} &\textbf{ 0.3} &\textbf{ 0.7} &\textbf{-0.3} &\textbf{ 0.2} &\textbf{ 0.6} &$\mathbf{<1.69}$\\\hline
\end{tabular}
\caption[Limit variation as a function of bin size, $2lss$ channel.]{Limit variation as a function of bin size in the same-sign dimuon channel. (In bold: the final bin borders used in the $2lss$ channel.)}
\label{bin_limits_2lss}
\end{table}

The expected limit was found to be \textbf{r<1.69} for optimized bin borders in 10 bins and optimized pre-selection cuts.

\section{Other binning strategies}
Two additional strategies of clustering regions in the 2D plane of $BDTG_{tt}$ vs $BDTG_{ttV}$ into bins were attempted, following studies done and documented in great detail in Reference~\cite{CMS_AN_2017-029}. A brief description is provided in the following.

\textbf{Clustering by S/B ratio}\\
In this method, the 2D plane is clustered into a given number of bins corresponding to regions where S/B is within a certain range. The bin borders are determined such that the number of background events in each bin is approximately equal. The resulting regions for $2lss$ and $3l$  events are shown in Figure ~\ref{fig:sbbinning}, while the expected distribution of signal and dominant backgrounds are shown in Figure~\ref{fig:sbfinalbins}.

\begin{figure} [!h]
  \centering
  \includegraphics[width=0.45\textwidth]{binning/hTargetBinning_2lss.png}
  \includegraphics[width=0.45\textwidth]{binning/hTargetBinning_3l.png}
  \caption{Binning by S/B regions for $2lss$ (left) and $3l$ (right).}
  \label{fig:sbbinning}
\end{figure}

\begin{figure} [!h]
  \centering
  \includegraphics[width=0.45\textwidth]{binning/likelihoodBased_1d_2lss.pdf}
  \includegraphics[width=0.45\textwidth]{binning/likelihoodBased_1d_3l.pdf}
  \caption[Final bins (corresponding to S/B regions in the 2D plane)]{Final bins (corresponding to S/B regions in the 2D plane) for $2lss$ and $3l$ (right).}
  \label{fig:sbfinalbins}
\end{figure}

Using this method, the resulting limits (for the $\Ct=-1, \CV=1$ scenario) are about 20\% worse than with the binning in Section \ref{sec:binopt}: \mumu\ changed from 1.82 to 2.15, $3l$ changed from 1.52 to 1.75.

\textbf{$k$-Means geometric clustering}\\
This method employs a recursive application of the $k$-means algorithm (see Appendix D in Reference~\cite{CMS_AN_2017-029}) to separate the 2D plane into geometric regions. The resulting clustering (using the \ttH\ multilepton code on \tHq\ signal and \ttbar\ and \ttV\ background events) is shown in Figure ~\ref{fig:kmeansbinning}. The expected distribution of events for the signal and dominant backgrounds in these bins is shown in Fig.~\ref{fig:kmeansfinalbins}.
\begin{figure} [!h]
  \centering
  \includegraphics[width=0.45\textwidth]{binning/voronoi_2l_trial0.png}
  \includegraphics[width=0.45\textwidth]{binning/voronoi_3l_trial0.png}
  \caption[Binning into geometric regions using a $k$-means algorithm.]{Binning into geometric regions using a $k$-means algorithm for $2lss$ (left) and $3l$ (right).}
  \label{fig:kmeansbinning}
\end{figure}

\begin{figure} [!h]
  \centering
  \includegraphics[width=0.45\textwidth]{binning/recursiveNoOrdering_2l_trial0.png}
  \includegraphics[width=0.45\textwidth]{binning/recursiveNoOrdering_3l_trial0.png}
  \caption[Final bins using a $k$-means algorithm.]{Final bins using a $k$-means algorithm for $2lss$ (left) and $3l$ (right). Note that the bin numbering here is such that signal-like bins are lower.}
  \label{fig:kmeansfinalbins}
\end{figure}

Similarly to the S/B ratio binning, the limits using the $k$-means clustering are significantly worse than those of the bins described before. In the \mumu\ channel, the limit deteriorates from 1.82 to 2.05, whereas in $3l$ it changes from 1.58 to 1.78.

%\chapter{BDTG output variation with \CV and \Ct }\label{sec:bdtvscvct}

The BDTG classifier output was described in Section \label{secc:signal_disc} in the \Ct=-1,\CV=1 scenario; the change of BDTG classifiers output shape when varying the \CV/\Ct\ coupling scenario is shown in Figure~\ref{fig:bdtvscvct} in the $3l$ channel for five different values of \Ct, with \CV\ fixed at $1.0$.
\begin{figure} [!h]
  \centering
  \includegraphics[width=0.45\textwidth]{controlplots/bdtvscvct/thqMVA_ttv_3l.pdf}
  \includegraphics[width=0.45\textwidth]{controlplots/bdtvscvct/thqMVA_tt_3l.pdf} \\
  \caption[BDTG output variation with \CV/\Ct]{Change of the BDTG classifiers output when varying \Ct\ coupling (\CV\ is fixed at $1.0$). Training vs.\ \ttV\ (right) and vs.\ \ttbar\ (left).}
  \label{fig:bdtvscvct}
\end{figure}

Given that the BDT classifier output shape does not change, it is enought to train the BDTG in one of the \Ct/\CV points. It was chosen the SM point.  


%\chapter{\tHq-\ttH overlap}\label{app:overlap}

This section provides a quick overview of the differences and commonalities in event selections between this analysis and the \ttH\ multilepton search~\cite{CMS_AN_2017-029}. The object selections of the two analysis are perfectly synchronized due to shared frameworks and samples. The only exception is the usage of forward jets ($|\eta|>2.4, \pt>40$ GeV) in this analysis. Such jets are not considered in the \ttH\ analysis.

Table~\ref{tab:seldiffs} gives an overview of the main differences in the event selections. Here, $E^{miss}_{T\quad\text{LD}}$ is defined as $\MET\times0.00397 + H_{T}^{miss}\times0.00265$. Untagged jets in the \tHq\ analysis are jets that do not pass the CSV loose working point and are either central ($|\eta<2.4|, \pt>25$ GeV) or forward ($|\eta<2.4|, \pt>40$ GeV). All jets in the \ttH\ analysis are selected with $\pt>25$ GeV. Lepton $\pt$ cuts and the trigger selections are identical.

\begin{table}[h!]
\centering
\begin{tabular}{l|cc}
	Channel & \tHq & \ttH \\ \hline
	3l   & Z veto, 15 GeV\ & Z veto, 10 GeV\\\ 
	     & $N_\text{jets}^\text{b, med.}\geq1$ &
	       $N_\text{jets}^\text{b, med.}\geq1$ OR
	       $N_\text{jets}^\text{b, loose}\geq2$ \\\ 
	     & $\geq1$ un-tagged jet & $E^{miss}_{T\quad\text{LD}}> 0.2$ OR $N_\text{jets}^\text{centrl.}\geq4$ \\ \hline
	2lss & $N_\text{jets}^\text{b, med.}\geq1$ &
	       $N_\text{jets}^\text{b, med.}\geq1$ OR
	       $N_\text{jets}^\text{b, loose}\geq2$ \\\ 
	     & $\geq1$ un-tagged jet & $N_\text{jets}^\text{central}\geq4$ \\
\end{tabular}
\caption[Differences in event selection \tHq-\ttH\ multilepton analysis.]{Differences in event selection between this analysis and the \ttH\ multilepton analysis.}\label{tab:seldiffs}
\end{table}

Table~\ref{tab:overlap} shows the total event yields in the individual channels, and the yield of shared events between each channel, for the \tHq\ signal sample, the \ttH\ signal sample, and the data.
In the data, for the $3l$ channel, about $80\%$ of events passing the \tHq\ selection also pass the \ttH\ selection, constituting about $70\%$ of that channel. In the $2lss$ channel, about $50\%$ of data events passing the \tHq\ selection also pass the \ttH\ selection, but these events constitute almost $90\%$ of the \ttH\ selection in those channels. Similar overlaps are also seen in the \tHq\ and \ttH\ signal samples.

There is no migration between different channels and different selections, \ie, no events passing the selection of a given \tHq\ channel pass the selection of any other channels of \ttH\ and vice versa.

\begin{table}[h!]
\centering
% \begin{tabular}{cc|cccc}
% \textbf{\tHq\ sample} &       &               &              &             & \\\hline
%                       & Total & $\mumu(\ttH)$ & $\emu(\ttH)$ & \ee($\ttH$) & \threel($\ttH$)\\
% Total                 &       & 2353          & 3600         & 1106        & 2923              \\\hline
% $\mumu$(tHq)          & 7400  & 2166          & 0            & 0           & 0                 \\
% $\emu$(tHq)           & 11158 & 0             & 3321         & 0           & 0                 \\
% \ee(tHq)              & 3550  & 0             & 0            & 1025        & 0                  \\
% \threel(tHq)          & 3115  & 0             & 0            & 0           & 2347              \\\hline
% \textbf{\ttH\ sample} &       &               &              &             & \\\hline
%                       & Total & $\mumu(\ttH)$ & $\emu(\ttH)$ & \ee($\ttH$) & \threel($\ttH$)\\
% Total                 &       & 28703         & 42521        & 12869       & 30598             \\\hline
% $\mumu$(tHq)          & 32612 & 26547         & 0            & 0           & 0                 \\
% $\emu$(tHq)           & 48088 & 0             & 39164        & 0           & 0                 \\
% \ee(tHq)              & 15476 & 0             & 0            & 11896       & 0                 \\
% \threel(tHq)          & 26627 & 0             & 0            & 0           & 25288             \\\hline
% \textbf{Data}         &       &               &              &             & \\\hline
%                       & Total & $\mumu(\ttH)$ & $\emu(\ttH)$ & \ee($\ttH$) & \threel($\ttH$)\\
% Total                 &       & 160           & 280          & 90          & 154               \\\hline
% $\mumu$(tHq)          & 280   & 140           & 0            & 0           & 0                 \\
% $\emu$(tHq)           & 525   & 0             & 242          & 0           & 0                 \\
% \ee(tHq)              & 208   & 0             & 0            & 79          & 0                 \\
% \threel(tHq)          & 126   & 0             & 0            & 0           & 104               \\
\begin{tabular}{lrrrrr}
\textbf{\tHq\ sample} & \tHq\ & \ttH\ & Common & (\% \tHq) & (\% \ttH) \\ \hline
\mumu\                & 7400  & 2353  & 2166   & 29.3         & 92.1  \\
\emu\                 & 11158 & 3600  & 3321   & 29.8         & 92.2  \\
\ee\                  & 3550  & 1106  & 1025   & 28.9         & 92.7  \\
\threel\              & 3115  & 2923  & 2347   & 75.3         & 80.3  \\
 & & & & & \\
\textbf{\ttH\ sample} & \tHq\ & \ttH\ & Common & (\% \tHq) & (\% \ttH) \\ \hline
\mumu\                & 32612 & 28703 & 26547  & 81.4         & 92.5    \\
\emu\                 & 48088 & 42521 & 39164  & 81.4         & 92.1    \\
\ee\                  & 15476 & 12869 & 11896  & 76.9         & 92.4    \\
\threel\              & 26627 & 30598 & 25288  & 95.0         & 82.6    \\
 & & & & & \\
\textbf{Data}         & \tHq\ & \ttH\ & Common & (\% \tHq) & (\% \ttH) \\ \hline
\mumu\                & 280   & 160   & 140    & 50.0         & 87.5  \\
\emu\                 & 525   & 280   & 242    & 46.1         & 86.4  \\
\ee\                  & 208   & 90    & 79     & 38.0         & 87.8  \\
\threel\              & 126   & 154   & 104    & 82.5         & 67.5  \\

\end{tabular}
\caption[Individual and shared event yields \tHq-\ttH\ multilepton selections.]{Individual and shared event yields between this analysis (\tHq) and \ttH\ multilepton selections. }
\label{tab:overlap}
\end{table}

%\chapter{Forward jet impact plots}\label{app:forward_jet_impact}

The impact of the data/MC disagreement for forward jet $\eta$ is observed to be reduced with higher $\pt$ cuts. With a cut of 25 GeV in the \pt of the forward jet, the forward jet nuisance have the biggest impact in the fit (it is in the first place in Figure~\ref{fig:impact25}); when the \pt cut is increased to 30 GeV and 40 GeV, there is a reduction in the impact of the forward jet \etac nuisance in the fit as shown in Figures ~\ref{fig:impact30} and ~\ref{fig:impact40}. 

\begin{figure} [!h]
 \centering
 \includegraphics[width=1.0\textwidth]{limits/impacts/impacts_25.pdf}\\
\caption[Post-fit pulls and impacts with $\pt$ cut $25$ GeV for the forward jet]{Post-fit pulls and impacts of the 20 nuisance parameters with $\pt$ cut of $25$ GeV for the forward jet.}
\label{fig:impact25}
\end{figure}

\begin{figure} [!h]
 \centering
 \includegraphics[width=1.0\textwidth]{limits/impacts/impacts_30.pdf}\\
\caption[Post-fit pulls and impacts with $\pt$ cut $30$ GeV for the forward jet]{Post-fit pulls and impacts of the 20 nuisance parameters with $\pt$ cut of $30$ GeV for the forward jet.}
\label{fig:impact30}
\end{figure}

\begin{figure} [!h]
 \centering
 \includegraphics[width=1.0\textwidth]{limits/impacts/impacts_40.pdf}\\
\caption[Post-fit pulls and impacts with $\pt$ cut $25$ GeV for the forward jet]{Post-fit pulls and impacts of the 20 nuisance parameters with $\pt$ cut of $40$ GeV for the forward jet.}
\label{fig:impact40}
\end{figure}

%\chapter{Cross section and branching ratio scalings}\label{sec:xsbrscalings}

\begin{table}[h!]
  \centering
  \footnotesize
  \begin{tabular}{ll rrrrrrrrr}\hline
   \CV\ & \Ct\   & HWW    & HZZ    & H$\tau\tau$& H$\mu\mu$ & Hbb & Hcc & H$\gamma\gamma$ & H$Z\gamma$ & Hgg \\ \hline
   0.5  & -6.0   & 0.0827 & 0.0827 & 11.9098 & 11.9098 & 0.3308 & 0.3308 & 0.3308 & 0.3308 & 0.3308 \\
   0.5  & -4.0   & 0.1417 & 0.1417 & 9.0699  & 9.0699  & 0.5669 & 0.5669 & 0.5669 & 0.5669 & 0.5669 \\
   0.5  & -3.0   & 0.1889 & 0.1889 & 6.7999  & 6.7999  & 0.7555 & 0.7555 & 0.7555 & 0.7555 & 0.7555 \\
   0.5  & -2.5   & 0.2173 & 0.2173 & 5.4325  & 5.4325  & 0.8692 & 0.8692 & 0.8692 & 0.8692 & 0.8692 \\
   0.5  & -2.0   & 0.2478 & 0.2478 & 3.9647  & 3.9647  & 0.9912 & 0.9912 & 0.9912 & 0.9912 & 0.9912 \\
   0.5  & -1.5   & 0.2782 & 0.2782 & 2.5034  & 2.5034  & 1.1126 & 1.1126 & 1.1126 & 1.1126 & 1.1126 \\
   0.5  & -1.333 & 0.2877 & 0.2877 & 2.0448  & 2.0448  & 1.1508 & 1.1508 & 1.1508 & 1.1508 & 1.1508 \\
   0.5  & -1.25  & 0.2922 & 0.2922 & 1.8264  & 1.8264  & 1.1689 & 1.1689 & 1.1689 & 1.1689 & 1.1689 \\
   0.5  & -1.0   & 0.3048 & 0.3048 & 1.2194  & 1.2194  & 1.2194 & 1.2194 & 1.2194 & 1.2194 & 1.2194 \\
   0.5  & -0.833 & 0.3122 & 0.3122 & 0.8665  & 0.8665  & 1.2487 & 1.2487 & 1.2487 & 1.2487 & 1.2487 \\
   0.5  & -0.75  & 0.3154 & 0.3154 & 0.7097  & 0.7097  & 1.2617 & 1.2617 & 1.2617 & 1.2617 & 1.2617 \\
   0.5  & -0.667 & 0.3184 & 0.3184 & 0.5666  & 0.5666  & 1.2736 & 1.2736 & 1.2736 & 1.2736 & 1.2736 \\
   0.5  & -0.5   & 0.3235 & 0.3235 & 0.3235  & 0.3235  & 1.2938 & 1.2938 & 1.2938 & 1.2938 & 1.2938 \\
   0.5  & -0.333 & 0.3272 & 0.3272 & 0.1451  & 0.1451  & 1.3087 & 1.3087 & 1.3087 & 1.3087 & 1.3087 \\
   0.5  & -0.25  & 0.3285 & 0.3285 & 0.0821  & 0.0821  & 1.3139 & 1.3139 & 1.3139 & 1.3139 & 1.3139 \\
   0.5  & -0.167 & 0.3294 & 0.3294 & 0.0367  & 0.0367  & 1.3177 & 1.3177 & 1.3177 & 1.3177 & 1.3177 \\
   0.5  & 0.0    & 0.3302 & 0.3302 & 0.0000  & 0.0000  & 1.3207 & 1.3207 & 1.3207 & 1.3207 & 1.3207 \\
   0.5  & 0.167  & 0.3294 & 0.3294 & 0.0367  & 0.0367  & 1.3177 & 1.3177 & 1.3177 & 1.3177 & 1.3177 \\
   0.5  & 0.25   & 0.3285 & 0.3285 & 0.0821  & 0.0821  & 1.3139 & 1.3139 & 1.3139 & 1.3139 & 1.3139 \\
   0.5  & 0.333  & 0.3272 & 0.3272 & 0.1451  & 0.1451  & 1.3087 & 1.3087 & 1.3087 & 1.3087 & 1.3087 \\
   0.5  & 0.5    & 0.3235 & 0.3235 & 0.3235  & 0.3235  & 1.2938 & 1.2938 & 1.2938 & 1.2938 & 1.2938 \\
   0.5  & 0.667  & 0.3184 & 0.3184 & 0.5666  & 0.5666  & 1.2736 & 1.2736 & 1.2736 & 1.2736 & 1.2736 \\
   0.5  & 0.75   & 0.3154 & 0.3154 & 0.7097  & 0.7097  & 1.2617 & 1.2617 & 1.2617 & 1.2617 & 1.2617 \\
   0.5  & 0.833  & 0.3122 & 0.3122 & 0.8665  & 0.8665  & 1.2487 & 1.2487 & 1.2487 & 1.2487 & 1.2487 \\
   0.5  & 1.0    & 0.3048 & 0.3048 & 1.2194  & 1.2194  & 1.2194 & 1.2194 & 1.2194 & 1.2194 & 1.2194 \\
   0.5  & 1.25   & 0.2922 & 0.2922 & 1.8264  & 1.8264  & 1.1689 & 1.1689 & 1.1689 & 1.1689 & 1.1689 \\
   0.5  & 1.333  & 0.2877 & 0.2877 & 2.0448  & 2.0448  & 1.1508 & 1.1508 & 1.1508 & 1.1508 & 1.1508 \\
   0.5  & 1.5    & 0.2782 & 0.2782 & 2.5034  & 2.5034  & 1.1126 & 1.1126 & 1.1126 & 1.1126 & 1.1126 \\
   0.5  & 2.0    & 0.2478 & 0.2478 & 3.9647  & 3.9647  & 0.9912 & 0.9912 & 0.9912 & 0.9912 & 0.9912 \\
   0.5  & 2.5    & 0.2173 & 0.2173 & 5.4325  & 5.4325  & 0.8692 & 0.8692 & 0.8692 & 0.8692 & 0.8692 \\
   0.5  & 3.0    & 0.1889 & 0.1889 & 6.7999  & 6.7999  & 0.7555 & 0.7555 & 0.7555 & 0.7555 & 0.7555 \\
   0.5  & 4.0    & 0.1417 & 0.1417 & 9.0699  & 9.0699  & 0.5669 & 0.5669 & 0.5669 & 0.5669 & 0.5669 \\
   0.5  & 6.0    & 0.0827 & 0.0827 & 11.9098 & 11.9098 & 0.3308 & 0.3308 & 0.3308 & 0.3308 & 0.3308 \\\hline
    \end{tabular}
    \caption[Scalings of Higgs decay branching ratios vs.\ \Ct\ and \CV=0.5\ ]{Scalings of Higgs decay branching ratios vs.\ \Ct\ and \CV=0.5.}\label{tab:brscalingK6_0p5}
 \end{table}

\begin{table}[h!]
  \centering
  \footnotesize
  \begin{tabular}{ll rrrrrrrrr}\hline
   \CV\ & \Ct\   & HWW    & HZZ    & H$\tau\tau$& H$\mu\mu$ & Hbb    & Hcc    & H$\gamma\gamma$ & H$Z\gamma$ & Hgg \\ \hline
   1.0  & -6.0   & 0.3122 & 0.3122 & 11.2408    & 11.2408   & 0.3122 & 0.3122 & 0.3122          & 0.3122     & 0.3122 \\
   1.0  & -4.0   & 0.5144 & 0.5144 & 8.2305     & 8.2305    & 0.5144 & 0.5144 & 0.5144          & 0.5144     & 0.5144 \\
   1.0  & -3.0   & 0.6651 & 0.6651 & 5.9862     & 5.9862    & 0.6651 & 0.6651 & 0.6651          & 0.6651     & 0.6651 \\
   1.0  & -2.5   & 0.7517 & 0.7517 & 4.6979     & 4.6979    & 0.7517 & 0.7517 & 0.7517          & 0.7517     & 0.7517 \\
   1.0  & -2.0   & 0.8412 & 0.8412 & 3.3647     & 3.3647    & 0.8412 & 0.8412 & 0.8412          & 0.8412     & 0.8412 \\
   1.0  & -1.5   & 0.9271 & 0.9271 & 2.0859     & 2.0859    & 0.9271 & 0.9271 & 0.9271          & 0.9271     & 0.9271 \\
   1.0  & -1.333 & 0.9534 & 0.9534 & 1.6941     & 1.6941    & 0.9534 & 0.9534 & 0.9534          & 0.9534     & 0.9534 \\
   1.0  & -1.25  & 0.9658 & 0.9658 & 1.5091     & 1.5091    & 0.9658 & 0.9658 & 0.9658          & 0.9658     & 0.9658 \\
   1.0  & -1.0   & 1.0000 & 1.0000 & 1.0000     & 1.0000    & 1.0000 & 1.0000 & 1.0000          & 1.0000     & 1.0000 \\
   1.0  & -0.833 & 1.0196 & 1.0196 & 0.7075     & 0.7075    & 1.0196 & 1.0196 & 1.0196          & 1.0196     & 1.0196 \\
   1.0  & -0.75  & 1.0283 & 1.0283 & 0.5784     & 0.5784    & 1.0283 & 1.0283 & 1.0283          & 1.0283     & 1.0283 \\
   1.0  & -0.667 & 1.0362 & 1.0362 & 0.4610     & 0.4610    & 1.0362 & 1.0362 & 1.0362          & 1.0362     & 1.0362 \\
   1.0  & -0.5   & 1.0495 & 1.0495 & 0.2624     & 0.2624    & 1.0495 & 1.0495 & 1.0495          & 1.0495     & 1.0495 \\
   1.0  & -0.333 & 1.0593 & 1.0593 & 0.1175     & 0.1175    & 1.0593 & 1.0593 & 1.0593          & 1.0593     & 1.0593 \\
   1.0  & -0.25  & 1.0627 & 1.0627 & 0.0664     & 0.0664    & 1.0627 & 1.0627 & 1.0627          & 1.0627     & 1.0627 \\
   1.0  & -0.167 & 1.0652 & 1.0652 & 0.0297     & 0.0297    & 1.0652 & 1.0652 & 1.0652          & 1.0652     & 1.0652 \\
   1.0  & 0.0    & 1.0672 & 1.0672 & 0.0000     & 0.0000    & 1.0672 & 1.0672 & 1.0672          & 1.0672     & 1.0672 \\
   1.0  & 0.167  & 1.0652 & 1.0652 & 0.0297     & 0.0297    & 1.0652 & 1.0652 & 1.0652          & 1.0652     & 1.0652 \\
   1.0  & 0.25   & 1.0627 & 1.0627 & 0.0664     & 0.0664    & 1.0627 & 1.0627 & 1.0627          & 1.0627     & 1.0627 \\
   1.0  & 0.333  & 1.0593 & 1.0593 & 0.1175     & 0.1175    & 1.0593 & 1.0593 & 1.0593          & 1.0593     & 1.0593 \\
   1.0  & 0.5    & 1.0495 & 1.0495 & 0.2624     & 0.2624    & 1.0495 & 1.0495 & 1.0495          & 1.0495     & 1.0495 \\
   1.0  & 0.667  & 1.0362 & 1.0362 & 0.4610     & 0.4610    & 1.0362 & 1.0362 & 1.0362          & 1.0362     & 1.0362 \\
   1.0  & 0.75   & 1.0283 & 1.0283 & 0.5784     & 0.5784    & 1.0283 & 1.0283 & 1.0283          & 1.0283     & 1.0283 \\
   1.0  &  0.833 & 1.0196 & 1.0196 & 0.7075     & 0.7075    & 1.0196 & 1.0196 & 1.0196          & 1.0196     & 1.0196 \\
   1.0  & 1.0    & 1.0000 & 1.0000 & 1.0000     & 1.0000    & 1.0000 & 1.0000 & 1.0000          & 1.0000     & 1.0000 \\
   1.0  & 1.25   & 0.9658 & 0.9658 & 1.5091     & 1.5091    & 0.9658 & 0.9658 & 0.9658          & 0.9658     & 0.9658 \\
   1.0  & 1.333  & 0.9534 & 0.9534 & 1.6941     & 1.6941    & 0.9534 & 0.9534 & 0.9534          & 0.9534     & 0.9534 \\
   1.0  & 1.5    & 0.9271 & 0.9271 & 2.0859     & 2.0859    & 0.9271 & 0.9271 & 0.9271          & 0.9271     & 0.9271 \\
   1.0  & 2.0    & 0.8412 & 0.8412 & 3.3647     & 3.3647    & 0.8412 & 0.8412 & 0.8412          & 0.8412     & 0.8412 \\
   1.0  & 2.5    & 0.7517 & 0.7517 & 4.6979     & 4.6979    & 0.7517 & 0.7517 & 0.7517          & 0.7517     & 0.7517 \\
   1.0  & 3.0    & 0.6651 & 0.6651 & 5.9862     & 5.9862    & 0.6651 & 0.6651 & 0.6651          & 0.6651     & 0.6651 \\
   1.0  & 4.0    & 0.5144 & 0.5144 & 8.2305     & 8.2305    & 0.5144 & 0.5144 & 0.5144          & 0.5144     & 0.5144 \\
   1.0  & 6.0    & 0.3122 & 0.3122 & 11.2408    & 11.2408   & 0.3122 & 0.3122 & 0.3122          & 0.3122     & 0.3122 \\\hline
    \end{tabular}
    \caption[Scalings of Higgs decay branching ratios vs.\ \Ct\ and \CV=1.0 ]{Scalings of Higgs decay branching ratios vs.\ \Ct\ and \CV=1.0.}\label{tab:brscalingK6_1}
 \end{table}

\begin{table}[h!]
  \centering
  \footnotesize
  \begin{tabular}{ll rrrrrrrrr}\hline
   \CV\ & \Ct\   & HWW    & HZZ    & H$\tau\tau$& H$\mu\mu$ & Hbb & Hcc & H$\gamma\gamma$ & H$Z\gamma$ & Hgg \\ \hline
   1.5  & -6.0   & 0.6424 & 0.6424 & 10.2785 & 10.2785 & 0.2855 & 0.2855 & 0.2855 & 0.2855 & 0.2855 \\
   1.5  & -4.0   & 1.0028 & 1.0028 & 7.1307  & 7.1307  & 0.4457 & 0.4457 & 0.4457 & 0.4457 & 0.4457 \\
   1.5  & -3.0   & 1.2477 & 1.2477 & 4.9909  & 4.9909  & 0.5545 & 0.5545 & 0.5545 & 0.5545 & 0.5545 \\
   1.5  & -2.5   & 1.3802 & 1.3802 & 3.8338  & 3.8338  & 0.6134 & 0.6134 & 0.6134 & 0.6134 & 0.6134 \\
   1.5  & -2.0   & 1.5115 & 1.5115 & 2.6870  & 2.6870  & 0.6718 & 0.6718 & 0.6718 & 0.6718 & 0.6718 \\
   1.5  & -1.5   & 1.6322 & 1.6322 & 1.6322  & 1.6322  & 0.7254 & 0.7254 & 0.7254 & 0.7254 & 0.7254 \\
   1.5  & -1.333 & 1.6682 & 1.6682 & 1.3175  & 1.3175  & 0.7414 & 0.7414 & 0.7414 & 0.7414 & 0.7414 \\
   1.5  & -1.25  & 1.6851 & 1.6851 & 1.1702  & 1.1702  & 0.7489 & 0.7489 & 0.7489 & 0.7489 & 0.7489 \\
   1.5  & -1.0   & 1.7310 & 1.7310 & 0.7693  & 0.7693  & 0.7693 & 0.7693 & 0.7693 & 0.7693 & 0.7693 \\
   1.5  & -0.833 & 1.7570 & 1.7570 & 0.5419  & 0.5419  & 0.7809 & 0.7809 & 0.7809 & 0.7809 & 0.7809 \\
   1.5  & -0.75  & 1.7684 & 1.7684 & 0.4421  & 0.4421  & 0.7860 & 0.7860 & 0.7860 & 0.7860 & 0.7860 \\
   1.5  & -0.667 & 1.7788 & 1.7788 & 0.3517  & 0.3517  & 0.7906 & 0.7906 & 0.7906 & 0.7906 & 0.7906 \\
   1.5  & -0.5   & 1.7962 & 1.7962 & 0.1996  & 0.1996  & 0.7983 & 0.7983 & 0.7983 & 0.7983 & 0.7983 \\
   1.5  & -0.333 & 1.8089 & 1.8089 & 0.0891  & 0.0891  & 0.8039 & 0.8039 & 0.8039 & 0.8039 & 0.8039 \\
   1.5  & -0.25  & 1.8133 & 1.8133 & 0.0504  & 0.0504  & 0.8059 & 0.8059 & 0.8059 & 0.8059 & 0.8059 \\
   1.5  & -0.167 & 1.8165 & 1.8165 & 0.0225  & 0.0225  & 0.8073 & 0.8073 & 0.8073 & 0.8073 & 0.8073 \\
   1.5  & 0.0    & 1.8191 & 1.8191 & 0.0000  & 0.0000  & 0.8085 & 0.8085 & 0.8085 & 0.8085 & 0.8085 \\
   1.5  & 0.167  & 1.8165 & 1.8165 & 0.0225  & 0.0225  & 0.8073 & 0.8073 & 0.8073 & 0.8073 & 0.8073 \\
   1.5  & 0.25   & 1.8133 & 1.8133 & 0.0504  & 0.0504  & 0.8059 & 0.8059 & 0.8059 & 0.8059 & 0.8059 \\
   1.5  & 0.333  & 1.8089 & 1.8089 & 0.0891  & 0.0891  & 0.8039 & 0.8039 & 0.8039 & 0.8039 & 0.8039 \\
   1.5  & 0.5    & 1.7962 & 1.7962 & 0.1996  & 0.1996  & 0.7983 & 0.7983 & 0.7983 & 0.7983 & 0.7983 \\
   1.5  & 0.667  & 1.7788 & 1.7788 & 0.3517  & 0.3517  & 0.7906 & 0.7906 & 0.7906 & 0.7906 & 0.7906 \\
   1.5  & 0.75   & 1.7684 & 1.7684 & 0.4421  & 0.4421  & 0.7860 & 0.7860 & 0.7860 & 0.7860 & 0.7860 \\
   1.5  & 0.833  & 1.7570 & 1.7570 & 0.5419  & 0.5419  & 0.7809 & 0.7809 & 0.7809 & 0.7809 & 0.7809 \\
   1.5  & 1.0    & 1.7310 & 1.7310 & 0.7693  & 0.7693  & 0.7693 & 0.7693 & 0.7693 & 0.7693 & 0.7693 \\
   1.5  & 1.25   & 1.6851 & 1.6851 & 1.1702  & 1.1702  & 0.7489 & 0.7489 & 0.7489 & 0.7489 & 0.7489 \\
   1.5  & 1.333  & 1.6682 & 1.6682 & 1.3175  & 1.3175  & 0.7414 & 0.7414 & 0.7414 & 0.7414 & 0.7414 \\
   1.5  & 1.5    & 1.6322 & 1.6322 & 1.6322  & 1.6322  & 0.7254 & 0.7254 & 0.7254 & 0.7254 & 0.7254 \\
   1.5  & 2.0    & 1.5115 & 1.5115 & 2.6870  & 2.6870  & 0.6718 & 0.6718 & 0.6718 & 0.6718 & 0.6718 \\
   1.5  & 2.5    & 1.3802 & 1.3802 & 3.8338  & 3.8338  & 0.6134 & 0.6134 & 0.6134 & 0.6134 & 0.6134 \\
   1.5  & 3.0    & 1.2477 & 1.2477 & 4.9909  & 4.9909  & 0.5545 & 0.5545 & 0.5545 & 0.5545 & 0.5545 \\
   1.5  & 4.0    & 1.0028 & 1.0028 & 7.1307  & 7.1307  & 0.4457 & 0.4457 & 0.4457 & 0.4457 & 0.4457 \\
   1.5  & 6.0    & 0.6424 & 0.6424 & 10.2785 & 10.2785 & 0.2855 & 0.2855 & 0.2855 & 0.2855 & 0.2855 \\\hline
    \end{tabular}
    \caption[Scalings of Higgs decay branching ratios vs.\ \Ct\ and \CV=1.5]{Scalings of Higgs decay branching ratios vs.\ \Ct\ and \CV=1.5.}\label{tab:brscalingK6_1p5}
 \end{table}
\begin{landscape}
\begin{table}[h!]                                                                                                                                                                          
  \centering                                                                                                                                                                               
  \footnotesize                                                                                                                                                                            
  \begin{tabular}{ll rrr rrr rrr}\hline                                                                                                                                                          
   \CV\ & \Ct\  & ttHWW  & ttHZZ  & ttH$\tau\tau$& tHqWW & tHqZZ & tHq$\tau\tau$& tHWWW & tHWZZ & tHW$\tau\tau$ \\ \hline   
   0.5 & -6.0   & 2.9775 & 2.9775 & 428.7530 & 9.2066 & 9.2066 & 1325.7460 & 9.7660 & 9.7660 & 1406.3049 \\
   0.5 & -4.0   & 2.2675 & 2.2675 & 145.1182 & 7.5740 & 7.5740 & 484.7357  & 7.8819 & 7.8819 & 504.4411 \\
   0.5 & -3.0   & 1.7000 & 1.7000 & 61.1988  & 6.1214 & 6.1214 & 220.3702  & 6.2562 & 6.2562 & 225.2227 \\
   0.5 & -2.5   & 1.3581 & 1.3581 & 33.9529  & 5.1857 & 5.1857 & 129.6430  & 5.2277 & 5.2277 & 130.6931 \\
   0.5 & -2.0   & 0.9912 & 0.9912 & 15.8589  & 4.1227 & 4.1227 & 65.9633   & 4.0762 & 4.0762 & 65.2197 \\
   0.5 & -1.5   & 0.6259 & 0.6259 & 5.6327   & 2.9838 & 2.9838 & 26.8544   & 2.8645 & 2.8645 & 25.7805  \\
   0.5 & -1.333 & 0.5112 & 0.5112 & 3.6333   & 2.6025 & 2.6025 & 18.4974   & 2.4648 & 2.4648 & 17.5190 \\
   0.5 & -1.25  & 0.4566 & 0.4566 & 2.8538   & 2.4154 & 2.4154 & 15.0962   & 2.2700 & 2.2700 & 14.1878 \\
   0.5 & -1.0   & 0.3048 & 0.3048 & 1.2194   & 1.8696 & 1.8696 & 7.4784    & 1.7078 & 1.7078 & 6.8310 \\
   0.5 & -0.833 & 0.2166 & 0.2166 & 0.6012   & 1.5271 & 1.5271 & 4.2386    & 1.3605 & 1.3605 & 3.7760 \\
   0.5 & -0.75  & 0.1774 & 0.1774 & 0.3992   & 1.3657 & 1.3657 & 3.0729    & 1.1987 & 1.1987 & 2.6970 \\
   0.5 & -0.667 & 0.1417 & 0.1417 & 0.2521   & 1.2111 & 1.2111 & 2.1553    & 1.0451 & 1.0451 & 1.8598 \\
   0.5 & -0.5   & 0.0809 & 0.0809 & 0.0809   & 0.9236 & 0.9236 & 0.9236    & 0.7640 & 0.7640 & 0.7640 \\
   0.5 & -0.333 & 0.0363 & 0.0363 & 0.0161   & 0.6720 & 0.6720 & 0.2981    & 0.5249 & 0.5249 & 0.2328 \\
   0.5 & -0.25  & 0.0205 & 0.0205 & 0.0051   & 0.5618 & 0.5618 & 0.1405    & 0.4231 & 0.4231 & 0.1058 \\
   0.5 & -0.167 & 0.0092 & 0.0092 & 0.0010   & 0.4622 & 0.4622 & 0.0516    & 0.3334 & 0.3334 & 0.0372 \\
   0.5 & 0.0    & 0.0000 & 0.0000 & 0.0000   & 0.2953 & 0.2953 & 0.0000    & 0.1909 & 0.1909 & 0.0000 \\
   0.5 & 0.167  & 0.0092 & 0.0092 & 0.0010   & 0.1755 & 0.1755 & 0.0196    & 0.1010 & 0.1010 & 0.0113 \\
   0.5 & 0.25   & 0.0205 & 0.0205 & 0.0051   & 0.1339 & 0.1339 & 0.0335    & 0.0762 & 0.0762 & 0.0191 \\
   0.5 & 0.333  & 0.0363 & 0.0363 & 0.0161   & 0.1043 & 0.1043 & 0.0463    & 0.0647 & 0.0647 & 0.0287 \\
   0.5 & 0.5    & 0.0809 & 0.0809 & 0.0809   & 0.0809 & 0.0809 & 0.0809    & 0.0809 & 0.0809 & 0.0809 \\
   0.5 & 0.667  & 0.1417 & 0.1417 & 0.2521   & 0.1044 & 0.1044 & 0.1859    & 0.1480 & 0.1480 & 0.2634 \\
   0.5 & 0.75   & 0.1774 & 0.1774 & 0.3992   & 0.1329 & 0.1329 & 0.2991    & 0.1993 & 0.1993 & 0.4485 \\
   0.5 & 0.833  & 0.2166 & 0.2166 & 0.6012   & 0.1720 & 0.1720 & 0.4775    & 0.2620 & 0.2620 & 0.7272 \\
   0.5 & 1.0    & 0.3048 & 0.3048 & 1.2194   & 0.2811 & 0.2811 & 1.1243    & 0.4200 & 0.4200 & 1.6801 \\
   0.5 & 1.25   & 0.4566 & 0.4566 & 2.8538   & 0.5119 & 0.5119 & 3.1993    & 0.7270 & 0.7270 & 4.5438 \\
   0.5 & 1.333  & 0.5112 & 0.5112 & 3.6333   & 0.6041 & 0.6041 & 4.2939    & 0.8449 & 0.8449 & 6.0051 \\
   0.5 & 1.5    & 0.6259 & 0.6259 & 5.6327   & 0.8096 & 0.8096 & 7.2863    & 1.1020 & 1.1020 & 9.9179 \\
   0.5 & 2.0    & 0.9912 & 0.9912 & 15.8589  & 1.5402 & 1.5402 & 24.6428   & 1.9827 & 1.9827 & 31.7238 \\
   0.5 & 2.5    & 1.3581 & 1.3581 & 33.9529  & 2.3549 & 2.3549 & 58.8716   & 2.9329 & 2.9329 & 73.3233 \\
   0.5 & 3.0    & 1.7000 & 1.7000 & 61.1988  & 3.1686 & 3.1686 & 114.0678  & 3.8625 & 3.8625 & 139.0502 \\
   0.5 & 4.0    & 2.2675 & 2.2675 & 145.1182 & 4.6200 & 4.6200 & 295.6829  & 5.4873 & 5.4873 & 351.1881 \\
   0.5 & 6.0    & 2.9775 & 2.9775 & 428.7530 & 6.6207 & 6.6207 & 953.3740  & 7.6698 & 7.6698 & 1104.4467 \\\hline
  \end{tabular}
  \caption[Scalings of $\sigma\times$BR for the signal components and \CV=0.5\ ]{Scalings of cross section times BR, for the different \ttH, \tHq, \tHW\ signal components and \CV=0.5\ .}\label{tab:xsbrscalingK6_0p5}
\end{table}

\begin{table}[h!]
  \centering
  \footnotesize
  \begin{tabular}{ll rrr rrr rrr}\hline
   \CV\ & \Ct\  & ttHWW  & ttHZZ  & ttH$\tau\tau$& tHqWW & tHqZZ & tHq$\tau\tau$& tHWWW & tHWZZ & tHW$\tau\tau$ \\ \hline
   1.0 & -6.0   & 11.2408 & 11.2408 & 404.6686 & 40.4768 & 40.4768 & 1457.1666 & 41.3681 & 41.3681 & 1489.2533 \\
   1.0 & -4.0   & 8.2305  & 8.2305  & 131.6886 & 34.2339 & 34.2339 & 547.7422  & 33.8480 & 33.8480 & 541.5676 \\
   1.0 & -3.0   & 5.9862  & 5.9862  & 53.8759  & 28.5396 & 28.5396 & 256.8562  & 27.3983 & 27.3983 & 246.5850 \\
   1.0 & -2.5   & 4.6979  & 4.6979  & 29.3616  & 24.8511 & 24.8511 & 155.3195  & 23.3557 & 23.3557 & 145.9734 \\
   1.0 & -2.0   & 3.3647  & 3.3647  & 13.4590  & 20.6360 & 20.6360 & 82.5440   & 18.8497 & 18.8497 & 75.3987 \\
   1.0 & -1.5   & 2.0859  & 2.0859  & 4.6933   & 16.0557 & 16.0557 & 36.1254   & 14.0919 & 14.0919 & 31.7068 \\
   1.0 & -1.333 & 1.6941  & 1.6941  & 3.0102   & 14.4942 & 14.4942 & 25.7545   & 12.5059 & 12.5059 & 22.2216 \\
   1.0 & -1.25  & 1.5091  & 1.5091  & 2.3579   & 13.7201 & 13.7201 & 21.4377   & 11.7273 & 11.7273 & 18.3239 \\
   1.0 & -1.0   & 1.0000  & 1.0000  & 1.0000   & 11.4220 & 11.4220 & 11.4220   & 9.4484  & 9.4484  & 9.4484 \\
   1.0 & -0.833 & 0.7075  & 0.7075  & 0.4909   & 9.9372  & 9.9372  & 6.8953    & 8.0059  & 8.0059  & 5.5552 \\
   1.0 & -0.75  & 0.5784  & 0.5784  & 0.3254   & 9.2212  & 9.2212  & 5.1869    & 7.3200  & 7.3200  & 4.1175 \\
   1.0 & -0.667 & 0.4610  & 0.4610  & 0.2051   & 8.5229  & 8.5229  & 3.7917    & 6.6579  & 6.6579  & 2.9620 \\
   1.0 & -0.5   & 0.2624  & 0.2624  & 0.0656   & 7.1807  & 7.1807  & 1.7952    & 5.4076  & 5.4076  & 1.3519 \\
   1.0 & -0.333 & 0.1175  & 0.1175  & 0.0130   & 5.9375  & 5.9375  & 0.6584    & 4.2814  & 4.2814  & 0.4748 \\
   1.0 & -0.25  & 0.0664  & 0.0664  & 0.0042   & 5.3616  & 5.3616  & 0.3351    & 3.7730  & 3.7730  & 0.2358 \\
   1.0 & -0.167 & 0.0297  & 0.0297  & 0.0008   & 4.8163  & 4.8163  & 0.1343    & 3.3009  & 3.3009  & 0.0921 \\
   1.0 & 0.0    & 0.0000  & 0.0000  & 0.0000   & 3.8183  & 3.8183  & 0.0000    & 2.4676  & 2.4676  & 0.0000 \\
   1.0 & 0.167  & 0.0297  & 0.0297  & 0.0008   & 2.9624  & 2.9624  & 0.0826    & 1.7981  & 1.7981  & 0.0501 \\
   1.0 & 0.25   & 0.0664  & 0.0664  & 0.0042   & 2.5928  & 2.5928  & 0.1620    & 1.5284  & 1.5284  & 0.0955 \\
   1.0 & 0.333  & 0.1175  & 0.1175  & 0.0130   & 2.2612  & 2.2612  & 0.2507    & 1.3014  & 1.3014  & 0.1443 \\
   1.0 & 0.5    & 0.2624  & 0.2624  & 0.0656   & 1.7115  & 1.7115  & 0.4279    & 0.9742  & 0.9742  & 0.2435 \\
   1.0 & 0.667  & 0.4610  & 0.4610  & 0.2051   & 1.3198  & 1.3198  & 0.5871    & 0.8188  & 0.8188  & 0.3643 \\
   1.0 & 0.75   & 0.5784  & 0.5784  & 0.3254   & 1.1834  & 1.1834  & 0.6657    & 0.8042  & 0.8042  & 0.4524 \\
   1.0 & 0.833  & 0.7075  & 0.7075  & 0.4909   & 1.0852  & 1.0852  & 0.7530    & 0.8301  & 0.8301  & 0.5760 \\
   1.0 & 1.0    & 1.0000  & 1.0000  & 1.0000   & 1.0000  & 1.0000  & 1.0000    & 1.0000  & 1.0000  & 1.0000 \\
   1.0 & 1.25   & 1.5091  & 1.5091  & 2.3579   & 1.1380  & 1.1380  & 1.7782    & 1.5278  & 1.5278  & 2.3872 \\
   1.0 & 1.333  & 1.6941  & 1.6941  & 3.0102   & 1.2492  & 1.2492  & 2.2197    & 1.7691  & 1.7691  & 3.1434 \\
   1.0 & 1.5    & 2.0859  & 2.0859  & 4.6933   & 1.5628  & 1.5628  & 3.5163    & 2.3434  & 2.3434  & 5.2727 \\
   1.0 & 2.0    & 3.3647  & 3.3647  & 13.4590  & 3.1023  & 3.1023  & 12.4092   & 4.6362  & 4.6362  & 18.5449 \\
   1.0 & 2.5    & 4.6979  & 4.6979  & 29.3616  & 5.2667  & 5.2667  & 32.9167   & 7.4799  & 7.4799  & 46.7493 \\
   1.0 & 3.0    & 5.9862  & 5.9862  & 53.8759  & 7.7435  & 7.7435  & 69.6914   & 10.5403 & 10.5403 & 94.8625 \\
   1.0 & 4.0    & 8.2305  & 8.2305  & 131.6886 & 12.7892 & 12.7892 & 204.6276  & 16.4642 & 16.4642 & 263.4266 \\
   1.0 & 6.0    & 11.2408 & 11.2408 & 404.6686 & 20.9516 & 20.9516 & 754.2573  & 25.5403 & 25.5403 & 919.4497 \\\hline
  \end{tabular}
  \caption[Scalings of $\sigma\times$BR for the signal components and \CV=1.0\ ]{Scalings of cross section times BR, for the different \ttH, \tHq, \tHW\ signal components and \CV=1.0\ .}\label{tab:xsbrscalingK6_1}
\end{table}

\begin{table}[h!]
  \centering
  \footnotesize
  \begin{tabular}{ll rrr rrr rrr}\hline
   \CV\ & \Ct\  & ttHWW   & ttHZZ & ttH$\tau\tau$& tHqWW   & tHqZZ & tHq$\tau\tau$& tHWWW & tHWZZ & tHW$\tau\tau$ \\ \hline
   1.5 & -6.0   & 23.1266 & 23.1266 & 370.0260   & 96.1923 & 96.1923 & 1539.0768  & 95.1080 & 95.1080 & 1521.7272 \\
   1.5 & -4.0   & 16.0441 & 16.0441 & 114.0913   & 81.6690 & 81.6690 & 580.7570   & 77.3512 & 77.3512 & 550.0531 \\
   1.5 & -3.0   & 11.2295 & 11.2295 & 44.9178    & 68.8703 & 68.8703 & 275.4812   & 62.9086 & 62.9086 & 251.6344 \\
   1.5 & -2.5   & 8.6261  & 8.6261  & 23.9614    & 60.7939 & 60.7939 & 168.8720   & 54.1622 & 54.1622 & 150.4505 \\
   1.5 & -2.0   & 6.0458  & 6.0458  & 10.7481    & 51.7152 & 51.7152 & 91.9381    & 44.6227 & 44.6227 & 79.3293 \\
   1.5 & -1.5   & 3.6725  & 3.6725  & 3.6725     & 41.9469 & 41.9469 & 41.9469    & 34.6991 & 34.6991 & 34.6991 \\
   1.5 & -1.333 & 2.9643  & 2.9643  & 2.3410     & 38.6171 & 38.6171 & 30.4971    & 31.4016 & 31.4016 & 24.7987 \\
   1.5 & -1.25  & 2.6330  & 2.6330  & 1.8284     & 36.9629 & 36.9629 & 25.6687    & 29.7807 & 29.7807 & 20.6810 \\
   1.5 & -1.0   & 1.7310  & 1.7310  & 0.7693     & 32.0233 & 32.0233 & 14.2326    & 25.0144 & 25.0144 & 11.1175 \\
   1.5 & -0.833 & 1.2192  & 1.2192  & 0.3760     & 28.7953 & 28.7953 & 8.8803     & 21.9653 & 21.9653 & 6.7740 \\
   1.5 & -0.75  & 0.9948  & 0.9948  & 0.2487     & 27.2234 & 27.2234 & 6.8058     & 20.5014 & 20.5014 & 5.1254 \\
   1.5 & -0.667 & 0.7914  & 0.7914  & 0.1565     & 25.6778 & 25.6778 & 5.0772     & 19.0767 & 19.0767 & 3.7720 \\
   1.5 & -0.5   & 0.4491  & 0.4491  & 0.0499     & 22.6628 & 22.6628 & 2.5181     & 16.3435 & 16.3435 & 1.8159 \\
   1.5 & -0.333 & 0.2006  & 0.2006  & 0.0099     & 19.7986 & 19.7986 & 0.9758     & 13.8117 & 13.8117 & 0.6807 \\
   1.5 & -0.25  & 0.1133  & 0.1133  & 0.0031     & 18.4397 & 18.4397 & 0.5122     & 12.6364 & 12.6364 & 0.3510 \\
   1.5 & -0.167 & 0.0507  & 0.0507  & 0.0006     & 17.1281 & 17.1281 & 0.2123     & 11.5203 & 11.5203 & 0.1428 \\
   1.5 & 0.0    & 0.0000  & 0.0000  & 0.0000     & 14.6443 & 14.6443 & 0.0000     & 9.4640  & 9.4640  & 0.0000 \\
   1.5 & 0.167  & 0.0507  & 0.0507  & 0.0006     & 12.3858 & 12.3858 & 0.1535     & 7.6760  & 7.6760  & 0.0951 \\
   1.5 & 0.25   & 0.1133  & 0.1133  & 0.0031     & 11.3529 & 11.3529 & 0.3154     & 6.8916  & 6.8916  & 0.1914 \\
   1.5 & 0.333  & 0.2006  & 0.2006  & 0.0099     & 10.3820 & 10.3820 & 0.5117     & 6.1783  & 6.1783  & 0.3045 \\
   1.5 & 0.5    & 0.4491  & 0.4491  & 0.0499     & 8.6227  & 8.6227  & 0.9581     & 4.9621  & 4.9621  & 0.5513 \\
   1.5 & 0.667  & 0.7914  & 0.7914  & 0.1565     & 7.1299  & 7.1299  & 1.4098     & 4.0411  & 4.0411  & 0.7990 \\
   1.5 & 0.75   & 0.9948  & 0.9948  & 0.2487     & 6.4888  & 6.4888  & 1.6222     & 3.6932  & 3.6932  & 0.9233 \\
   1.5 & 0.833  & 1.2192  & 1.2192  & 0.3760     & 5.9148  & 5.9148  & 1.8241     & 3.4176  & 3.4176  & 1.0540 \\
   1.5 & 1.0    & 1.7310  & 1.7310  & 0.7693     & 4.9627  & 4.9627  & 2.2057     & 3.0782  & 3.0782  & 1.3681 \\
   1.5 & 1.25   & 2.6330  & 2.6330  & 1.8284     & 4.0340  & 4.0340  & 2.8014     & 3.0873  & 3.0873  & 2.1440 \\
   1.5 & 1.333  & 2.9643  & 2.9643  & 2.3410     & 3.8531  & 3.8531  & 3.0429     & 3.2206  & 3.2206  & 2.5434 \\
   1.5 & 1.5    & 3.6725  & 3.6725  & 3.6725     & 3.6725  & 3.6725  & 3.6725     & 3.6725  & 3.6725  & 3.6725 \\
   1.5 & 2.0    & 6.0458  & 6.0458  & 10.7481    & 4.4580  & 4.4580  & 7.9254     & 6.3144  & 6.3144  & 11.2255 \\
   1.5 & 2.5    & 8.6261  & 8.6261  & 23.9614    & 6.8533  & 6.8533  & 19.0368    & 10.4359 & 10.4359 & 28.9887 \\
   1.5 & 3.0    & 11.2295 & 11.2295 & 44.9178    & 10.3536 & 10.3536 & 41.4143    & 15.4728 & 15.4728 & 61.8913 \\
   1.5 & 4.0    & 16.0441 & 16.0441 & 114.0913   & 18.9646 & 18.9646 & 134.8595   & 26.5208 & 26.5208 & 188.5926 \\
   1.5 & 6.0    & 23.1266 & 23.1266 & 370.0260   & 35.9359 & 35.9359 & 574.9741   & 46.2619 & 46.2619 & 740.1909 \\\hline
    \end{tabular}                                                                                                                                                                          
    \caption[Scalings of $\sigma\times$BR for the signal components and \CV=1.5\ ]{Scalings of cross section times BR, for the different \ttH, \tHq, \tHW\ signal components and \CV=1.5\ .}\label{tab:xsbrscalingK6_1p5}                              
 \end{table}   

\end{landscape}

\begin{table}[h!]
  \centering
  \footnotesize
  \begin{tabular}{l| ccc | cc}\hline
    $\cos(\alpha_{CP})$& Exp.                     & SM exp.                  & Obs.  & Best fit $\sigma [pb]$.  & Best fit r               \\ \hline
    -1.0             & $0.299^{0.130}_{-0.088}$ & $0.396^{0.190}_{-0.135}$ & 0.594 & $0.284^{0.183}_{-0.171}$ & $0.650^{0.418}_{-0.391}$ \\
    -0.9             & $0.297^{0.130}_{-0.088}$ & $0.388^{0.184}_{-0.132}$ & 0.578 & $0.268^{0.182}_{-0.171}$ & $0.686^{0.466}_{-0.438}$ \\
    -0.8             & $0.294^{0.129}_{-0.088}$ & $0.377^{0.179}_{-0.127}$ & 0.562 & $0.251^{0.181}_{-0.171}$ & $0.725^{0.522}_{-0.493}$ \\
    -0.7             & $0.292^{0.129}_{-0.087}$ & $0.377^{0.165}_{-0.132}$ & 0.545 & $0.235^{0.179}_{-0.170}$ & $0.768^{0.587}_{-0.556}$ \\
    -0.6             & $0.288^{0.128}_{-0.086}$ & $0.368^{0.155}_{-0.128}$ & 0.523 & $0.215^{0.177}_{-0.169}$ & $0.798^{0.658}_{-0.627}$ \\
    -0.5             & $0.285^{0.127}_{-0.086}$ & $0.365^{0.166}_{-0.132}$ & 0.500 & $0.194^{0.176}_{-0.167}$ & $0.813^{0.739}_{-0.701}$ \\
    -0.4             & $0.281^{0.126}_{-0.085}$ & $0.357^{0.150}_{-0.128}$ & 0.479 & $0.175^{0.174}_{-0.165}$ & $0.840^{0.833}_{-0.792}$ \\
    -0.3             & $0.279^{0.125}_{-0.084}$ & $0.350^{0.150}_{-0.125}$ & 0.463 & $0.162^{0.173}_{-0.162}$ & $0.884^{0.943}_{-0.884}$ \\
    -0.2             & $0.277^{0.124}_{-0.084}$ & $0.346^{0.153}_{-0.117}$ & 0.453 & $0.153^{0.172}_{-0.153}$ & $0.954^{1.068}_{-0.954}$ \\
    -0.1             & $0.277^{0.124}_{-0.084}$ & $0.345^{0.155}_{-0.123}$ & 0.454 & $0.154^{0.171}_{-0.154}$ & $1.075^{1.197}_{-1.075}$ \\
    0.0              & $0.279^{0.125}_{-0.084}$ & $0.353^{0.161}_{-0.130}$ & 0.469 & $0.167^{0.173}_{-0.164}$ & $1.304^{1.356}_{-1.282}$ \\
    0.1              & $0.285^{0.127}_{-0.086}$ & $0.371^{0.160}_{-0.137}$ & 0.504 & $0.197^{0.177}_{-0.167}$ & $1.683^{1.508}_{-1.427}$ \\
    0.2              & $0.293^{0.129}_{-0.087}$ & $0.390^{0.159}_{-0.143}$ & 0.556 & $0.246^{0.180}_{-0.171}$ & $2.234^{1.639}_{-1.552}$ \\
    0.3              & $0.300^{0.130}_{-0.089}$ & $0.416^{0.178}_{-0.152}$ & 0.610 & $0.303^{0.182}_{-0.171}$ & $2.860^{1.723}_{-1.612}$ \\
    0.4              & $0.302^{0.129}_{-0.088}$ & $0.422^{0.193}_{-0.143}$ & 0.644 & $0.349^{0.177}_{-0.166}$ & $3.331^{1.693}_{-1.587}$ \\
    0.5              & $0.296^{0.125}_{-0.086}$ & $0.434^{0.157}_{-0.145}$ & 0.651 & $0.374^{0.165}_{-0.159}$ & $3.452^{1.527}_{-1.467}$ \\
    0.6              & $0.284^{0.120}_{-0.082}$ & $0.425^{0.136}_{-0.141}$ & 0.639 & $0.377^{0.155}_{-0.150}$ & $3.261^{1.339}_{-1.298}$ \\
    0.7              & $0.270^{0.114}_{-0.078}$ & $0.408^{0.118}_{-0.133}$ & 0.616 & $0.366^{0.147}_{-0.140}$ & $2.910^{1.167}_{-1.111}$ \\
    0.8              & $0.258^{0.109}_{-0.074}$ & $0.386^{0.120}_{-0.120}$ & 0.594 & $0.354^{0.141}_{-0.132}$ & $2.530^{1.006}_{-0.945}$ \\
    0.9              & $0.246^{0.104}_{-0.071}$ & $0.358^{0.128}_{-0.105}$ & 0.570 & $0.341^{0.135}_{-0.126}$ & $2.161^{0.857}_{-0.798}$ \\
    1.0              & $0.238^{0.101}_{-0.069}$ & $0.351^{0.125}_{-0.101}$ & 0.555 & $0.331^{0.132}_{-0.121}$ & $1.851^{0.736}_{-0.679}$ \\\hline
  \end{tabular}
  \caption[Expected and observed upper limits for CP-mixing angles.]{Expected (for background only, and for a SM-like Higgs signal) and observed 95\% C.L. upper limits (in pb), and best fit signal strength $r$ a\
    nd corresponding best fit cross section for the combined $\tH+\ttH$ cross section times branching ratio for the combination of all three channels, for different values of $\cos(\alpha_{CP})$.}\label{tab:cp_xsec}
\end{table}


%% Appendices go here (if you have them)

%% Bibliography goes here (You better have one)
%% BibTeX is your friend
%\addcontentsline{toc}{chapter}{Bibliography}
%\bibliographystyle{abbrv}
%use following if all content of bibtex file should be shown
\nocite{*}
\begin{thebibliography}{300}

%%%%%%%%%%%%%%%%%%%%%%%%%%%%%%%%%%%%%%%%%%%%%%%
%%%%%%%%%%%%%%%%%%%    ch2    %%%%%%%%%%%%%%%%%
%%%%%%%%%%%%%%%%%%%%%%%%%%%%%%%%%%%%%%%%%%%%%%%

\bibitem{atlas}{The Atlas collaboration, Technical Design Report LHCC 99-14, ATLAS TDR 14. CERN, Geneva Switzerland, 1999}
\bibitem{cms_doc}{CMS Collaboration. The CMS Experiment at the CERN LHC. Volume 3, 2008.}
\bibitem{lhcb}{LHCb Collaboration, S. Amato et al., LHCb technical proposal,CERN-LHCC-98-04}
\bibitem{alice}{ALICE: Technical proposal for a large ion collider experiment at the CERN LHC, CERN-LHCC-95-71.}
\bibitem{accelproc} from \url{https://videos.cern.ch/record/1610170.}


%%%%%%%%%%%%%%%%%%%%%%%%%%%%%%%%%%%%%%%%%%%%%%%
%%%%%%%%%%%%%%%%    ch7    %%%%%%%%%%%%%%%%%%%%
%%%%%%%%%%%%%%%%%%%%%%%%%%%%%%%%%%%%%%%%%%%%%%%
\bibitem{pix_tdr} A. Dominguez et. al. ``CMS Technical Design Report for the Pixel Detector Upgrade'', CERN-LHCC-2012-016. CMS-TDR-11.
\bibitem{ph1_sop} UNL Phase I FPIX Assembly Standard Operating Procedures. From {https://github.com/psi46/unl-sop}, 2016.
%\bibitem{sop_106} UNL Phase I FPIX Assembly Standard Operating Procedures. From {https://github.com/psi46/unl-sop}, 2016.
\bibitem{and_the} J. Monroy. Search for production of a Higgs boson and a single top quark in multilepton final states in \MakeLowercase{pp} collisions at ${\sqrt{\MakeLowercase{s}}=\textrm{13 T\MakeLowercase{e}V}}$ July 2018.
\bibitem{gluing_frank} F. Meier `` Forward Pixel Module Glueing at UNL'' YouTube, Sep. 15, 2014, from \url{https://www.youtube.com/watch?v=ofdntTIwKY4}.
\bibitem{aerotech} Aerotech (n.d.). ``AGS15000 Series'', from \url{https://www.aerotech.com/product-catalog/gantry-system/ags15000.aspx?p=\%2fproduct-catalog\%2fgantry-system.aspx\%3f}
\bibitem{jmonroy_channel} J. Monroy (n.d). Home [YouTube Channel]. from \url{https://www.youtube.com/channel/UCi7S7vhYpieLOy2KJ0SS0eg}.
\bibitem{pxar} pXar software framework 2015, from \url{ https://twiki.cern.ch/twiki/bin/viewauth/CMS/Pxar.}
\bibitem{fpix_module_testing_guide} J. Antonelli. ``FPix Module Testing Reference Guide'', Sep. 2015, from \url{https://cms-docdb.cern.ch/cgi-bin/DocDB/RetrieveFile?docid=12690&filename=fpix-module-testing-reference-guide.pdf&version=5}


%\bibitem{Treses}
%\emph{C. H$\ddot{o}$rmann. Design and performance of the silicon pixel detector modules for the CMS experiment. Dissertation, Universit$\ddot{a}$ Z$\ddot{u}$rich, 2006}

%\bibitem{Doses}
%\emph{R. Horisberger. Status of pixel upgrade phase I. Presetation from Traker general meeting, CMS week, CERN 9 april 2013}



%\bibitem{Tres}
%\emph{The CMS collaboration, CMS Tracker, Technical Design Report LHCC 98-6, CERN, Geneva Switzerland, 1998}

%\bibitem{no used} {\rojo{ not used yet in the main document}
%\bibitem{pixel_performance} CMS Tracker Group. ``The Performance plots for Phase 1 Pixel Detector 2017'' \url{https://twiki.cern.ch/twiki/bin/view/CMSPublic/PixelOfflinePlotsAugust2017#Alignment_of_the_forward_pixels}, last accessed on 01.05.2018
%\bibitem{unl_sop} UNL Silicon pixel group ``Pixel Phase-I activities at University of Nebraska-Lincoln (UNL)''\url{https://twiki.cern.ch/twiki/bin/view/CMS/UNLPixelPhaseI}, last accessed on 01.05.2018.
%\bibitem{aerotech} Aerotech (n.d.). ``AGS15000 Series'', retrieved from \url{https://www.aerotech.com/product-catalog/gantry-system/ags15000.aspx?p=\%2fproduct-catalog\%2fgantry-system.aspx\%3f}
%\bibitem{pr_algorithm} C. Fangmeier. ``Fiducial recognition'', 2016. Github repository,\url{https://github.com/cfangmeier/Small/blob/master/JupyterNotebooks/Fiducial_Recognition.ipynb}
%\bibitem{sop_103} F. Meier and J. Monroy `` SOP 103 - Module assembly: Gluing of HDI to BBM'', retrieved from \url{https://twiki.cern.ch/twiki/pub/CMS/UNLPixelPhaseI/UNL-PxPhI-SOP-103-v0.pdf}, last accessed on 05.30.2018.
%\bibitem{gluing_frank} F. Meier `` Forward Pixel Module Glueing at UNL'' YouTube, Sep. 15, 2014, retrieved from \url{https://www.youtube.com/watch?v=ofdntTIwKY4}.  

%\bibitem{sop_105} F. Meier and J. Monroy `` SOP 105 - Module assembly: Encapsulation of wirebonds'', retrieved from \url{https://twiki.cern.ch/twiki/pub/CMS/UNLPixelPhaseI/UNL-PxPhI-SOP-103-v0.pdf}, last accessed on 05.30.2018.





%\bibitem{pixel_performance} CMS Tracker Group. ``The Performance plots for Phase 1 Pixel Detector 2017'' \url{https://twiki.cern.ch/twiki/bin/view/CMSPublic/PixelOfflinePlotsAugust2017#Alignment_of_the_forward_pixels}, last accessed on 01.05.2018











%\bibitem{maxwell} J.C. Maxwell. ``A dynamical theory of the electromagnetic field''. Philosophical Transactions of the Royal Society of London. 155: 459–512.(1865) doi:10.1098/rstl.1865.0008
%\bibitem{planck} M. Planck. ``Über das Gesetz der Energieverteilung im Normalspektrum''. Annalen der Physik. 4 (3): 553.(1901).
%\bibitem{photoeffect} A. Einstein ``Über einen die Erzeugung und Verwandlung des Lichtes betreffenden heuristischen Gesichtspunkt''. Annalen der Physik. 17 (6): 132–148, (1905).
%\bibitem{brownian} A. Einstein. ``Über die von der molekularkinetischen Theorie der Wärme geforderte Bewegung von in ruhenden Flüssigkeiten suspendierten Teilchen''. Annalen der Physik. 17 (8): 549–560, (1905).
%\bibitem{relativity} A. Einstein. ``Zur Elektrodynamik bewegter Körper''. Annalen der Physik. 17 (10): 891–921, (1905).
%\bibitem{energy} A. Einstein, ``Ist die Trägheit eines Körpers von seinem Energieinhalt abhängig?''. Annalen der Physik. 18 (13): 639–641, (1905).   
%\bibitem{ligo} B. P. Abbott et al. ``Observation of Gravitational Waves from a Binary Black Hole Merger''. PRL 116, 061102 (2016).
%\bibitem{schwinger} J. Schwinger. ``Quantum Electrodynamics. I. A Covariant Formulation''. Physical Review. 74 (10): 1439-61, (1948). \url{https://doi.org/10.1103/PhysRev.74.1439} 
%\bibitem{feynman} R. P. Feynman. ``Space-Time Approach to Quantum Electrodynamics''. Physical Review. 76 (6): 769-89, (1949). \url{https://doi.org/10.1103/PhysRev.76.769}
%\bibitem{tomonaga} S. Tomonaga. ``On a Relativistically Invariant Formulation of the Quantum Theory of Wave Fields''. Progress of Theoretical Physics. 1 (2): 27-42, (1946). \url{https://doi.org/10.1143/PTP.1.27}
%\bibitem{griffiths} D.J. Griffiths, ``Introduction to electrodynamics''. 4th ed. Pearson, (2013).
%\bibitem{mandl} F. Mandl, G. Shaw. ``Quantum field theory.'' Chichester, Wiley (2009).
%\bibitem{halzen} F. Halzen, and A.D. Martin, ``Quarks and leptons: An introductory course in modern particle physics''. New York: Wiley, (1984) .
%\bibitem{smpicture} File: Standard\_Model\_of\_Elementary\_Particle\_dark.svg. (2017, June 12) Wikimedia Commons, the free media repository. Retrieved November 27, 2017 from \url{https://www.collegiate-advanced-electricity.com/single-post/2017/04/10/The-Standard-Model-of-Particle-Physics}.
%\bibitem{noether}  E. Noether, ``Invariante Variationsprobleme'', Nachrichten von der Gesellschaft der Wissenschaften zu G\"ottingen, mathematisch-physikalische Klasse, vol. 1918, pp. 235-257, (1918).
%\bibitem{pdg} C. Patrignani et al. (Particle Data Group), Chin. Phys. C, 40, 100001 (2016) and 2017 update.  
%\bibitem{goldhaber} M. Goldhaber, L. Grodzins, A.W. Sunyar ``Helicity of Neutrinos'', Phys. Rev. 109, 1015 (1958).
%\bibitem{nu_mass} Palanque-Delabrouille N et al. ``Neutrino masses and cosmology with Lyman-alpha forest power spectrum'', JCAP 11 011 (2015).
%\bibitem{gellman} M. Gell-Mann. ``A Schematic Model of Baryons and Mesons''. Physics Letters. 8 (3): 214-215 (1964).
%\bibitem{zweig} G. Zweig. ``An SU(3) Model for Strong Interaction Symmetry and its Breaking'' (PDF). CERN Report No.8182/TH.401 (1964).
%\bibitem{zweig2} G. Zweig. ``An SU(3) Model for Strong Interaction Symmetry and its Breaking: II'' (PDF). CERN Report No.8419/TH.412(1964).
%\bibitem{gell_ni} M. Gell-Mann. ``The Interpretation of the New Particles as Displaced Charged Multiplets''. Il Nuovo Cimento 4: 848. (1956).
%\bibitem{gell_ni2} T. Nakano, K, Nishijima. ``Charge Independence for V-particles''. Progress of Theoretical Physics 10 (5): 581-582. (1953).  
%\bibitem{C} N. Cabibbo, ``Unitary symmetry and leptonic decays'' Physical Review Letters, vol. 10, no. 12, p. 531, (1963).
%\bibitem{KM} M.Kobayashi, T.Maskawa, ``CP-violation in the renormalizable theory of weak interaction,'' Progress of Theoretical Physics, vol. 49, no. 2, pp. 652-657, (1973).
%\bibitem{ckm} File: Weak Decay (flipped).svg. (2017, June 12). Wikimedia Commons, the free media repository. Retrieved November 27, 2017 from \url{https://commons.wikimedia.org/w/index.php?title=File:Weak\_Decay\_(flipped)\\.svg\&oldid=247498592}.
%\bibitem{hyperphys} Georgia Tech University. Coupling Constants for the Fundamental Forces(2005). Retrieved January 10, 2018, from \url{http://hyperphysics.phy-astr.gsu.edu/hbase/Forces/couple.html\#c2}
%\bibitem{matt} M. Strassler. (May 31, 2013).The Strengths of the Known Forces. Retrieved January 10, 2018, from \url{https://profmattstrassler.com/articles-and-posts/particle-physics-basics/the-known-forces-of-nature/the-strength-of-the-known-forces/}
%\bibitem{glashow} S.L. Glashow. ``Partial symmetries of weak interactions'', Nucl. Phys. 22 579-588, (1961).
%\bibitem{salam} A. Salam, J.C. Ward. ``Electromagnetic and weak interactions'', Physics Letters 13 168-171, (1964).
%\bibitem{weinberg} S. Weinberg, ``A model of leptons'', Physical Review Letters, vol. 19, no. 21, p. 1264, (1967).
%\bibitem{peskin} M. Peskin, D. Schroeder, ``An introduction to quantum field theory''. Perseus Books Publishing L.L.C., (1995).  
%\bibitem{pich} A. Pich. ``The Standard Model of Electroweak Interactions'' \url{https://arxiv.org/abs/1201.0537}
%\bibitem{wmass1}G. Arnison et al. (UA1 Collaboration), Phys. Lett. B 122, 103 (1983).
%\bibitem{wmass2} M. Banner et al. (UA2 Collaboration), Phys. Lett. B 122, 476 (1983).
%\bibitem{zmass1}G. Arnison et al. (UA1 Collaboration), Phys. Lett. B 126, 398 (1983).
%\bibitem{zmass2}P. Bagnaia et al. (UA2 Collaboration), Phys. Lett. B 129, 130 (1983).
%\bibitem{broken_symmetry} F.Bellaiche. (2012, 2 September). ``What's this Higgs boson anyway?''. Retrieved from: \url{https://www.quantum-bits.org/?p=233} 
%\bibitem{endres} M. Endres et al. Nature 487, 454-458 (2012) doi:10.1038/nature11255
%\bibitem{englert} F. Englert, R. Brout. ``Broken Symmetry and the Mass of Gauge Vector Mesons''. Physical Review Letters. 13 (9): 321-23.(1964) doi:10.1103/PhysRevLett.13.321
%\bibitem{higgs} P.Higgs. ``Broken Symmetries and the Masses of Gauge Bosons''. Physical Review Letters. 13 (16): 508-509,(1964). doi:10.1103/PhysRevLett.13.508.
%\bibitem{ghk} G.Guralnik, C.R. Hagen  and T.W.B. Kibble. ``Global Conservation Laws and Massless Particles''. Physical Review Letters. 13 (20): 585-587, (1964). doi:10.1103/PhysRevLett.13.585.
%\bibitem{hcms} CMS collaboration. ``Observation of a new boson at a mass of 125 GeV with the CMS experiment at the LHC''. Physics Letters B. 716 (1): 30-61 (2012). arXiv:1207.7235. doi:10.1016/j.physletb.2012.08.021
%\bibitem{hatlas} ATLAS collaboration. ``Observation of a New Particle in the Search for the Standard Model Higgs Boson with the ATLAS Detector at the LHC''. Physics Letters B. 716 (1): 1-29 (2012). arXiv:1207.7214.  doi:10.1016/j.physletb.2012.08.020.
%\bibitem{hmass} ATLAS collaboration; CMS collaboration (26 March 2015). ``Combined Measurement of the Higgs Boson Mass in pp Collisions at $\sqrt{s}=7$ and 8 TeV with the ATLAS and CMS Experiments''. Physical Review Letters. 114 (19): 191803. arXiv:1503.07589. doi:10.1103/PhysRevLett.114.191803.
%\bibitem{pp_coll} LHC\@ InternationalMasterclasses``When protons collide''. Retrieved from \url{http://atlas.physicsmasterclasses.org/en/zpath\_protoncollisions.htm}
%\bibitem{hcswg} CMS Collaboration, ``SM Higgs Branching Ratios and Total Decay Widths (up-date in CERN Report4 2016)''. \url{https://twiki.cern.ch/twiki/bin/view/LHCPhysics/CERNYellowReportPageBR} , last accessed on 17.12.2017.
%\bibitem{riley} R.Grant V. ``Determination of Higgs branching ratios in $H\to W^+W^-\to l\nu jj$ and $H\to ZZ \to l^+l^-jj$ channels''. Physics Department, University of Tennessee (Dated: October 31, 2012). Retrieved from \url{http://aesop.phys.utk.edu/ph611/2012/projects/Riley.pdf}
%\bibitem{denner} LHC Higgs Cross Section Working Group, Denner, A., Heinemeyer, S. et al. ``Standard model Higgs-boson branching ratios with uncertainties''. Eur. Phys. J. C (2011) 71: 1753. \url{https://doi.org/10.1140/epjc/s10052-011-1753-8}
%\bibitem{florian} D. de Florian et al., LHC Higgs Cross Section Working Group, CERN-2017-002-M, arXiv:1610.07922[hep-ph] (2016).
%\bibitem{comb_ht_couplings} ATLAS and CMS Collaborations, ``Measurements of the Higgs boson production and decay rates and constraints on its couplings from a combined ATLAS and CMS analysis of the LHC pp collision data at $\sqrt{s}$ = 7 and 8 TeV,'' (2016). CERN-EP-2016-100, ATLAS-HIGG-2015-07, CMS-HIG-15-002.  
%\bibitem{lhiggs} J. A. Aguilar-Saavedra, R. Benbrik, S. Heinemeyer, and M. Perez-Victoria, ``Handbook of vector-like quarks: Mixing and single production'', Phys. Rev. D 88 (2013) 094010, doi:10.1103/PhysRevD.88.094010, arXiv:1306.0572.
%\bibitem{greljo}A. Greljo, J. F. Kamenik, and J. Kopp, ``Disentangling flavor violation in the top-Higgs sector at the LHC'', JHEP 07 (2014) 046, doi:10.1007/JHEP07(2014)046, arXiv:1404.1278.
%\bibitem{maltoni2} F. Demartin, F. Maltoni, K. Mawatari, and M. Zaro, ``Higgs production in association with a single top quark at the LHC,'' European Physical Journal C, vol. 75, p. 267, (2015). doi:10.1140/epj
%c/s10052-015-3475-9, arXiv:1504.00611.
%\bibitem{demartin} F. Demartin, B. Maier, F. Maltoni, K. Mawatari, and M. Zaro, ``tWH associated production at the LHC'', European Physical Journal C, vol. 77, p. 34, (2017). arXiv:1607.05862
%\bibitem{maltoni1} F. Maltoni, K. Paul, T. Stelzer, and S. Willenbrock, ``Associated production of Higgs and single top at hadron colliders'', Phys.Rev. D64 (2001) 094023, [hep-ph/0106293].
%\bibitem{biswas} S. Biswas, E. Gabrielli, F. Margaroli, and B. Mele, ``Direct constraints on the top-Higgs coupling from the 8 TeV LHC data,'' Journal of High Energy Physics, vol. 07, p. 073, (2013).
%\bibitem{farina} M. Farina, C. Grojean, F. Maltoni, E. Salvioni, and A. Thamm, ``Lifting degeneracies in Higgs couplings using single top production in association with a Higgs boson,'' Journal of High Energy Physics, vol. 05, p. 022, (2013).
%\bibitem{tait} T.M. Tait and C.-P. Yuan, ``Single top quark production as a window to physics beyond the standard model'', Phys. Rev. D 63 (2000) 014018 [hep-ph/0007298].

%\bibitem{thqw_xsec} CMS Collaboration, ``Modelling of the single top-quark production in association with the Higgs boson at 13 TeV.'' \url{https://twiki.cern.ch/twiki/bin/viewauth/CMS/SingleTopHiggsGeneration13TeV}, last accessed on 16.01.2018.
%\bibitem{thb_xsec} CMS Collaboration, ``SM Higgs production cross sections at $\sqrt{s} = 13$ TeV.'' \url{https://twiki.cern.ch/twiki/bin/view/LHCPhysics/CERNYellowReportPageAt13TeV}, last accessed on 16.01.2018.
%\bibitem{dawson} S. Dawson, The effective W approximation, Nucl. Phys. B 249 (1985) 42.
%\bibitem{biswas2} S. Biswas, E. Gabrielli and B. Mele, JHEP 1301 (2013) 088 [arXiv:1211.0499 [hep-ph]].

%\bibitem{yellow} LHC Higgs Cross Section Working Group, ``Handbook of LHC Higgs Cross Sections: 4.Deciphering the Nature of the Higgs Sector'', arXiv:1610.07922.
%\bibitem{ellis} J. Ellis, D. S. Hwang, K. Sakurai, and M. Takeuchi.``Disentangling Higgs-Top Couplings in Associated Production'', JHEP 1404 (2014) 004, [arXiv:1312.5736].
%\bibitem{cms_ht_couplings} CMS Collaboration, V. Khachatryan et al., ``Precise determination of the mass of the Higgs boson and tests of compatibility of its couplings with the standard model predictions using proton collisions at 7 and 8 TeV,'' arXiv:1412.8662.
%\bibitem{atlas_ht_couplings} ATLAS Collaboration, G. Aad et al., ``Updated coupling measurements of the Higgs boson with the ATLAS detector using up to 25 fb$^-1$ of proton-proton collision data'', ATLAS-CONF-2014-009.


%%   %%%%%%%%%% references  CMS chapter%%%%%%%%%%%%

%\bibitem{cern} File:Cern-accelerator-complex.svg. Wikimedia Commons, the free media repository. Retrieved January, 2018 from \url{https://commons.wikimedia.org/wiki/File:Cern-accelerator-complex.svg}
%\bibitem{lep} J.L. Caron , ``Layout of the LEP tunnel including future LHC infrastructures.'', (Nov, 1993). A C Collection. Legacy of AC. Pictures from 1992 to 2002. Retrieved from \url{https://cds.cern.ch/record/841542}
%\bibitem{rfq} M. Vretenar, ``The radio-frequency quadrupole''. CERN Yellow Report CERN-2013-001, pp.207-223 DOI:10.5170/CERN-2013-001.207. arXiv:1303.6762
%\bibitem{lyndon} L.Evans. P. Bryant (editors). ``LHC Machine''. JINST 3 S08001 (2008).
%\bibitem{rfq2} CERN Photographic Service.``Radio-frequency quadrupole, RFQ-1'', March 1983, CERN-AC-8303511. Retrieved from \url{https://cds.cern.ch/record/615852}.
%\bibitem{video} CERN Photographic Service ``Animation of CERN's accelerator network'', 14 October 2013. DOI: 10.17181/cds.1610170 Retrieved from \url{https://videos.cern.ch/record/1610170}
%\bibitem{linac} C.Sutton. ``Particle accelerator''.Encyclopedia Britannica. July 17, 2013. Retrieved from \url{https://www.britannica.com/technology/particle-accelerator}.
%\bibitem{lhc_rfc} L.Guiraud. ``Installation of LHC cavity in vacuum tank.''. July 27 2000. CERN-AC-0007016. Retrieved from \url{https://cds.cern.ch/record/41567}.
%\bibitem{dipole_field} J.L. Caron, ``Magnetic field induced by the LHC dipole's superconducting coils''. March 1998. AC Collection. Legacy of AC. Pictures from 1992 to 2002. LHC-PHO-1998-325. Retrieved from \url{https://cds.cern.ch/record/841511}.
%\bibitem{lhc_dipole} AC Team. ``Diagram of an LHC dipole magnet''. June 1999. CERN-DI-9906025 retrieved from \url{https://cds.cern.ch/record/40524}. 
%\bibitem{lumi} CMS Collaboration ``Public CMS Luminosity Information''. \url{https://twiki.cern.ch/twiki/bin/view/CMSPublic/LumiPublicResults\#2016\\\_proton\_proton\_13\_TeV\_collis}, last accessed 24.01.2018
%\bibitem{lhc_layout} J.L Caron. ``LHC Layout'' AC Collection. Legacy of AC. Pictures from 1992 to 2002. September 1997, LHC-PHO-1997-060. Retrieved from \url{https://cds.cern.ch/record/841573}.
%\bibitem{pu}J.A. Coarasa. ``The CMS Online Cluster:Setup, Operation and Maintenance of an Evolving Cluster''. ISGC 2012, 26 February - 2 March 2012, Academia Sinica, Taipei, Taiwan.
%\bibitem{cms} CMS Collaboration. ``The CMS experiment at the CERN LHC'' JINST 3 S08004 (2008).
%\bibitem{cms_drawing} CMS Collaboration. ``CMS detector drawings 2012'' CMS-PHO-GEN-2012-002. Retrieved from \url{http://cds.cern.ch/record/1433717}.
%\bibitem{cms_slice} Davis, Siona Ruth. ``Interactive Slice of the CMS detector'', Aug. 2016, CMS-OUTREACH-2016-027, retrieved from \url{https://cds.cern.ch/record/2205172}  
%\bibitem{yoke1} M. Brice ``CMS central barrel yoke''. October 2002, CERN-EX-0210014. Retrieved from  \url{https://cds.cern.ch/record/43475
%\bibitem{yoke2} R. Breedon. ``View through the CMS detector during the cooldown of the solenoid on February 2006. CMS Collection'', February 2006, CMS-PHO-OREACH-2005-004, Retrieved from \url{https://cds.cern.ch/record/930094}.
%\bibitem{tracker} Halyo, V. and LeGresley, P. and Lujan, P. ``Massively Parallel Computing and the Search for Jets and Black Holes at the LHC'', Nucl.Instrum.Meth. A744 (2014) 54-60, DOI: 10.1016/j.nima.2014.01.038''
%\bibitem{pix_tdr} A. Dominguez et. al. ``CMS Technical Design Report for the Pixel Detector Upgrade'', CERN-LHCC-2012-016. CMS-TDR-11.
%\bibitem{sst} CMS Collaboration. ``Description and performance of track and primary-vertex reconstruction with the CMS tracker,'' Journal of Instrumentation, vol. 9, no. 10, p. P10009,(2014).
%\bibitem{tib} CMS Collaboration and M. Brice. ``Images of the CMS Tracker Inner Barrel'', November 2008, CMS-PHO-TRACKER-2008-002. Retrieved from \url{https://cds.cern.ch/record/1431467}.
%\bibitem{tec} M. Weber. ``The CMS tracker''. 6th international conference on hyperons, charm and beauty hadrons Chicago, June 28-July 3 2004.
%\bibitem{hcal} CMS Collaboration. ``Projected Performance of an Upgraded CMS Detector at the LHC and HL-LHC: Contribution to the Snowmass Process''. Jul 26, 2013. arXiv:1307.7135
%\bibitem{hb} L. Veillet. ``End assembly of HB with EB rails and rotation inside SX '',January 2002. CMS-PHO-HCAL-2002-002. Retrieved from \url{https://cds.cern.ch/record/42594}.
%\bibitem{muon} J. Puerta-Pelayo.``First DT+RPC chambers installation round in the UX5 cavern.''. January 2007, CMS-PHO-OREACH-2007-001. Retrieved from \url{https://cds.cern.ch/record/1019185}
%\bibitem{l1} X. Cid Vidal and R. Cid Manzano. ``CMS Global Muon Trigger'' web site: Taking a closer look at LHC. Retrieved from \url{https://www.lhc-closer.es/taking\_a\_closer\_look\_at\_lhc/0.lhc\_trigger}
%\bibitem{wlcg} WLCG Project Office, ``Documents \& Reference - Tiers - Structure,'' (2014). \url{http://wlcg.web.cern.ch/documents-reference} , last accessed on 30.01.2018.
%\bibitem{cmssw}CMS Collaboration. ``CMSSW Application Framework'', \url{https://twiki.cern.ch/twiki/bin/view/CMSPublic/WorkBookCMSSWFramework}, last accesses 06.02.2018 


  %%%%%%%%%%%%%%%%%%%%%%%references for gen sim reco charper%%%%%%%%%%%%%%%


%\bibitem{gen} A. Buckleya, J. Butterworthb, S. Giesekec, et. al. ``General-purpose event generators for LHC physics''. arXiv:1101.2599v1 [hep-ph] 13 Jan 2011
%\bibitem{gen_scheme} A. Quadt. ``Top Quark Physics at Hadron Colliders''. Advances in the Physics of Particles and Nuclei. Springer-Verlag Berlin Heidelberg. DOI: 10.1007/978-3-540-71060-8 (2007)
%\bibitem{pdfplot} DurhamHep Data Project, ``The Durham HepData Project - PDF Plotter.'' \url{http://hepdata.cedar.ac.uk/pdf/pdf3.html} , last accessed on 02.02.2018.
%\bibitem{dglap1} G. Altarelli and G. Parisi. ``ASYMPTOTIC FREEDOM IN PARTON LANGUAGE'', Nucl.Phys. B126:298 (1977).
%\bibitem{dglap2} Yu.L. Dokshitzer. Sov.Phys. JETP 46:641 (1977)
%\bibitem{dglap3} V.N. Gribov, L.N. Lipatov. ``Deep inelastic e p scattering in perturbation theory'', Sov.J.Nucl.Phys. 15:438 (1972)
%\bibitem{schemes}F. Maltoni, G. Ridolfi, and M. Ubiali, ``b-initiated processes at the LHC: a reappraisal,'' Journal of High Energy Physics, vol. 07, p. 022, (2012).
%\bibitem{lund}B. Andersson, G. Gustafson, G.Ingelman and T. Sjostrand, ``Parton fragmentation and string dynamics'', Physics Reports, Vol. 97, No. 2-3, pp. 31-145, 1983.
%\bibitem{tune}CMS Collaboration, ``Event generator tunes obtained from underlying event and multiparton scattering measurements;'' European Physical Journal C, vol. 76, no. 3, p. 155, (2016).
%\bibitem{madgraph} J. Alwall et. al., ``The automated computation of tree-level and next-to-leading order differential cross sections, and their matching to parton shower simulations,'' Journal of High Energy Physics, vol. 07, p. 079, (2014).
%\bibitem{pt_order} T. Sjöstrand and P. Z. Skands, ``Transverse-momentum-ordered showers and interleaved multiple interactions,'' European Physical Journal C, vol. 39, pp. 129–154, (2005).
%\bibitem{powheg}S. Frixione, P. Nason, and C. Oleari, ``Matching NLO QCD computations with Parton Shower simulations: the POWHEG method,'' Journal of High Energy Physics, vol. 11, p. 070, (2007).
%\bibitem{geant} S. Agostinelli et al., ``GEANT4: A Simulation toolkit,'' Nuclear Instruments and Methods in Physics, vol. A506, pp. 250–303, (2003).
%\bibitem{geant2} J.Allison et.al.,``Recent developments in Geant4'', Nuclear Instruments and Methods in Physics Research A 835 (2016) 186-225.  
%\bibitem{cmssim} CMS Collaboration ``Full Simulation Offline Guide'', \url{https://twiki.cern.ch/twiki/bin/view/CMSPublic/SWGuideSimulation}, last accessed 04.02.2018
%\bibitem{fastsim} A. Giammanco. ``The Fast Simulation of the CMS Experiment'' J. Phys.: Conf. Ser. 513 022012 (2014) 
%\bibitem{particle_flow} A.M. Sirunyan et. al. ``Particle-flow reconstruction and global event description with the CMS detector'', JINST 12 P10003 (2017) \url{https://doi.org/10.1088/1748-0221/12/10/P10003}.
%\bibitem{particle_flow2} CMS Collaboration. `` Description and performance of track and primary vertex reconstruction with the CMS tracker''. JINST 9 P10009 (2014). doi:10.1088/1748-0221/9/10/P10009
%\bibitem{pfdiag} J. Incandela. ``Status of the CMS SM Higgs Search'' July 4, 2012. Pdf slides. Retrieved from \url{https://indico.cern.ch/event/197461/contributions/1478917/attachments/290954/406673/CMS_4July2012_Final.pdf}
%\bibitem{ctf} P. Billoir and S. Qian, ``Simultaneous pattern recognition and track fitting by the Kalman filtering method'', Nucl. Instrum. Meth. A 294 219. (1990).
%\bibitem{gsf} W. Adam, R. Fruhwirth, A. Strandlie and T. Todorov, ``Reconstruction of electrons with the Gaussian sum filter in the CMS tracker at LHC'', eConf C 0303241 (2003) TULT009 [physics/0306087].
%\bibitem{da} K. Rose,  ``Deterministic Annealing for Clustering, Compression, Classification, Regression and related Optimisation Problems'', Proc. IEEE 86 (1998) 2210.
%\bibitem{avf} R. Fruhwirth, W. Waltenberger and P. Vanlaer, `` Adaptive Vertex Fitting'', CMS Note 2007-008 (2007).
%\bibitem{muon_req} CMS collaboration, ``Performance of CMS muon reconstruction in pp collision events at $\sqrt{s}= 7$ TeV '', JINST 7 P10002 2012, [arXiv:1206.4071].
%\bibitem{coco} Coco, Victor and Delsart, Pierre-Antoine and Rojo-Chacon, Juan and Soyez, Gregory and Sander, Christian, ``Jets and jet algorithms'', Proceedings, HERA and the LHC Workshop Series on the implications of HERA for LHC physics: 2006-2008, pag. 182-204. \url{http://inspirehep.net/record/866539/files/access.pdf}, (2009), doi:10.3204/DESY-PROC-2009-02/54
%\bibitem{antikt} M. Cacciari, G. P. Salam, and G. Soyez, ``The anti-$k_t$ jet clustering algorithm,'' Journal of High Energy Physics, vol. 04, p. 063, (2008).
%\bibitem{kt} S. Catani, Y. L. Dokshitzer, M. H. Seymour, and B. R. Webber, ``Longitudinally invariant $K_t$ clustering algorithms for hadron hadron collisions'', Nuclear Physics B, vol. 406, pp. 187–224, (1993).
%\bibitem{ac} Y.L. Dokshitzer, G.D. Leder, S.Moretti, and B.R. Webber, ``Better jet clustering algorithms,'' Journal of High Energy Physics, vol. 08, p. 001, (1997).
%\bibitem{jetconeview} B. Dorney. ``Anatomy of a Jet in CMS''. Quantum Diaries. June 1st, 2011.  Retrieved from \url{https://www.quantumdiaries.org/2011/06/01/anatomy-of-a-jet-in-cms/}
%\bibitem{jetreco} CMS Collaboration.``Event Displays from the high-energy collisions at 7 TeV'', May 2010, CMS-PHO-EVENTS-2010-007, Retrieved from \url{https://cds.cern.ch/record/1429614}.
%\bibitem{jec} CMS collaboration. ``Determination of jet energy calibration and transverse momentum resolution in CMS''. JINST 6 P11002 (2011). \url{http://dx.doi.org/10.1088/1748-0221/6/11/P11002}
%\bibitem{jec2} CMS Collaboration, ``Introduction to Jet Energy Corrections at CMS.''. \url{https://twiki.cern.ch/twiki/bin/view/CMS/IntroToJEC}, last accessed 10.02.2018. 
%\bibitem{btag}CMS Collaboration Collaboration. ``Identification of b quark jets at the CMS Experiment in the LHC Run 2''. Tech. rep. CMS-PAS-BTV-15-001. Geneva: CERN, (2016). \url{https://cds.cern.ch/record/2138504}.
%\bibitem{metcorr}CMS Collaboration Collaboration. ``Performance of missing energy reconstruction in 13 TeV pp collision data using the CMS detector''. Tech. rep. CMS-PAS-JME16-004. Geneva: CERN, 2016. \url{https://cds.cern.ch/record/2205284}.
%\bibitem{hig13}CMS Collaboration, ``New CMS results at Moriond (Electroweak) 2013'', Retrieved from \url{http://cms.web.cern.ch/sites/cms.web.cern.ch/files/styles/large/public/field/image/HIG13004\_Event01\_0.png?itok=LAwZzPHR}
%\bibitem{top12035} CMS Collaboration, ``New CMS results at Moriond (Electroweak) 2013'', Retrieved from \url{http://cms.web.cern.ch/sites/cms.web.cern.ch/files/styles/large/public/field/image/TOP12035\_Event01.png?itok=uMdnSqzC}
%\bibitem{cmsshow} CMS Collaboration. ``Physics Analysis Oriented Event Display ( Fireworks / cmsShow )'', \url{https://twiki.cern.ch/twiki/bin/view/CMSPublic/WorkBookFireworks}, last accessed 10.02.2018.

%% %%%%%%%%%%%%%%%%%%%%references for statistics chapter %%%%%%%%%%%%%%%%%%%%%%  

%\bibitem{mva} G. Cowan. ``Topics in statistical data analysis for high-energy physics''. arXiv:1012.3589v1
%\bibitem{tmva}  A. Hoecker et al., ``TMVA-Toolkit for multivariate data analysis'' \href{https://arxiv.org/pdf/physics/0703039.pdf}{arXiv:physics/0703039v5 (2009)} 
%\bibitem{luca} L. Lista. ``Statistical Methods for Data Analysis in Particle Physics'', 2nd ed. Springer International Publishing. (2017) \url{https://dx.doi.org/10.1007/978-3-319-62840-0}
%\bibitem{root} I. Antcheva et al., ``ROOT-A C++ framework for petabyte data storage, statistical analysis and visualization ,'' Computer Physics Communications, vol. 182, no. 6, pp. 1384–1385, (2011).
%\bibitem{coadou} Y. Coadou. ``Boosted decision trees'', ESIPAP, Archamps, 9 February 2016. Lecture. Retrieved from \url{https://indico.cern.ch/event/472305/contributions/1982360/attachments/1224979/1792797/ESIPAP_MVA160208-BDT.pdf}
%\bibitem{friedman} J.H. Friedman. ``Greedy function approximation: A gradient boosting machine''. Ann. Statist. Volume 29, Number 5 (2001), 1189-1232. \url{https://projecteuclid.org/download/pdf_1/euclid.aos/1013203451}.
%\bibitem{roofit} W. Verkerke and D. Kirkby, ``The RooFit toolkit for data modeling,'' arXiv preprint physics, (2003).
%\bibitem{combine} CMS Collaboration, ``Documentation of the RooStats-based statistics tools for Higgs PAG''. \url{https://twiki.cern.ch/twiki/bin/view/CMS/SWGuideHiggsAnalysisCombinedLimit}, last accessed on 08.04.2018.
%\bibitem{minuit}  F. James, M. Roos, ``MINUIT: Function minimization and error analysis''. Cern Computer Centre Program Library, Geneve Long Write-up No. D506, 1989
%\bibitem{npl} J. Neyman and E. S. Pearson, ``On the problem of the most efficient tests of statistical hypotheses''. Springer-Verlag, (1992).

%\bibitem{read} A.L. Read. ``Modified frequentist analysis of search results (the $CL_s$ method),'' (2000). CERN-OPEN-2000-205.
%\bibitem{hgg} C. Palmer. ``Searches for a Light Higgs with CMS'', CMS-CR-2012-215. \url{https://cds.cern.ch/record/1560435}.
%\bibitem{wald}A. Wald, ``Tests of statistical hypotheses concerning several parameters when the number of observations is large'', Transactions of the American Mathematical society, vol. 54, no. 3, pp. 426–482, (1943).
%\bibitem{asymptotic} G. Cowan, K. Cranmer, E. Gross, and O. Vitells, ``Asymptotic formulae for likelihood-based tests of new physics'', European Physical Journal C, vol. 71, p. 1554, (2011).
%\bibitem{wilks} S. S. Wilks, ``The Large-Sample Distribution of the Likelihood Ratio for Testing Composite Hypotheses'', Annals of Mathematical Statistics, vol. 9, pp. 60-62, (03, 1938).

  
  %%   %%%%%%%%%%%%%%%%%%%%references for Anlysis chapter %%%%%%%%%%%%%%%%%%%%%%
  
  
%\bibitem{h_mass} G. Aad et al. (ATLAS Collaboration, CMS Collaboration). ``Combined Measurement of the Higgs Boson Mass in pp Collisions at $\sqrt{s}=7$ and 8 TeV with the ATLAS and CMS Experiments''.  Phys. Rev. Lett. 114, 191803. \url{https://arxiv.org/pdf/1503.07589.pdf}.
%\bibitem{hespel} B. Hespel, F. Maltoni, and E. Vryonidou, ``Higgs and Z boson associated production via gluon fusion in the SM and the 2HDM'', JHEP 06 (2015) 065, \url{https://dx.doi:10.1007/JHEP06(2015)065}, arXiv:1503.01656.
%\bibitem{diboson} ATLAS Collaboration, ``Measurements of Higgs boson production and couplings in diboson final states with the ATLAS detector at the LHC'', Phys. Lett. B726 (2013) 88-119, doi:10.1016/j.physletb.2014.05.011,10.1016/j.physletb.2013.08.010, arXiv:1307.1427. [Erratum: Phys. Lett.B734,406(2014)].
%\bibitem{Khachatryan_2015}CMS Collaboration, ``Search for the associated production of a Higgs boson with a single top quark in proton-proton collisions at  $\sqrt{s}=8 $ TeV'', JHEP 06 (2016) 177,doi:10.1007/JHEP06(2016)177, arXiv:1509.08159.
%\bibitem{CMS_AN_2014-140}B. Stieger, C. Jorda Lope et al., ``Search for Associated Production of a Single Top Quark and a Higgs Boson in Leptonic Channels'', CMS Analysis Note CMS AN-14-140, 2014.
%\bibitem{CMS_AN_2016-211}M. Peruzzi, C. Mueller, B. Stieger et al., ``Search for ttH in multilepton final states at $\sqrt{s}=13$ TeV'', CMS Analysis Note CMS AN-16-211, 2016.
%\bibitem{CMS_PAS_HIG_16-019}CMS Collaboration, ``Search for H to bbar in association with a single top quark as a test of Higgs boson couplings at $\sqrt{s}=13$ TeV'', CMS Physics Analysis Summary CMS-PAS-HIG-16-019, 2016.
%\bibitem{CMS_PAS_HIG_17-005}CMS Collaboration, ``Search for production of a Higgs boson and a single top quark in multilepton final states in proton collisions at $\sqrt{s}=13$ TeV'', CMS Physics Analysis Summary CMS-PAS-HIG-17-005, 2016.
%\bibitem{dqm} CMS Collaboration. ``The CMS data quality monitoring software: experience and future prospects'', J. Phys.: Conf. Ser. 513 (2014) 032024

  
%\bibitem{json} CMS Collaboration, ``PdmV2016Analysis,'' (2016). \url{https://twiki.cern.ch/twiki/bin/viewauth/CMS/PdmV2016Analysis#DATA}, last accessed 11.04.2016.
%\bibitem{CMS_AN_2017-029} M. Peruzzi, F. Romeo, B. Stieger et al., ``Search for ttH in multilepton final states at $\sqrt{s}=13$ TeV'', CMS Analysis Note CMS AN-17-029, 2017.
%\bibitem{THQProdTwiki} B. Maier, ``SingleTopHiggProduction13TeV'', February, 2016. \url{https://twiki.cern.ch/twiki/bin/viewauth/CMS/SingleTopHiggsGeneration13TeV}.
%\bibitem{btag_corr} B. WG, ``BtagRecommendation80XReReco'', February, 2017. \url{https://twiki.cern.ch/twiki/bin/view/CMS/BtagRecommendation80XReReco}.
%\bibitem{btag_points} CMS Collaboration, ``Identification of b quark jets at the CMS Experiment in the LHC Run 2'', CMS Physics Analysis Summary CMS-PAS-BTV-15-001, 2016.
%\bibitem{muid} CMS Collaboration, ``Baseline muon selections for Run-II.'' \url{https://twiki.cern.ch/twiki/bin/viewauth/CMS/SWGuideMuonIdRun2}, last accessed on 24.02.2018.
%\bibitem{medium_muon} G. Petrucciani and C. Botta, ``Two step prompt muon identification'', January, 2015. \url{https://indico.cern.ch/event/368007/contribution/2/material/slides/0.pdf}.
%\bibitem{mva_eid1} CMS Collaboration, ``Performance of electron reconstruction and selection with the CMS detector in proton-proton collisions at $\sqrt{s} = 8$ TeV''. JINST 10 P06005, 2015. \url{https://doi.org/10.1088/1748-0221/10/06/P06005}
%\bibitem{mva_eid} H. Brun and C. Ochando, ``Updated Results on MVA eID with 13 TeV samples'', October, 2014. \url{https://indico.cern.ch/event/298249/contribution/3/material/slides/0.pdf}.
%\bibitem{i_mini} K. Rehermann and B. Tweedie, ``Efficient Identification of Boosted Semileptonic Top Quarks at the LHC'', JHEP 03 (2011) 059, \url{https://dx.doi:10.1007/JHEP03(2011)059}, arXiv:1007.2221.
%\bibitem{tnp} CMS Collaboration. ``Tag and Probe'', \url{https://twiki.cern.ch/twiki/bin/view/CMSPublic/TagAndProbe}, last accessed on 02.03.2018. 
%\bibitem{kcouplings} CMS Collaboration. ``$\kappa$ coupling modifiers'', \url{https://twiki.cern.ch/twiki/bin/view/LHCPhysics/LHCHXSWG2KAPPA#t_ch_qbtHq},last accessed on 27.04.2018.
%\bibitem{lumi} CMS Collaboration, ``CMS Luminosity Measurement for the 2016 Data Taking Period,'' (2017). CMS-PAS-LUM-17-001.
%\bibitem{CMS_AN_2016-378} J. Monroy, P.Das, B. Stieger et al., ``Search for tHq in multilepton final states at $\sqrt{s}=13$ TeV'', CMS Analysis Note CMS AN-16-378, 2016.
%\bibitem{fink} T. Chwalek 1 , N. Faltermann 1 , S. Fink 1 , et. al., ``Investigating the top-Yukawa coupling with the production of a Higgs boson in association with a single top quark in the $H \to b\bar{b}$ decay channel. CMS Analysis Note CMS AN-16-065, Version 4, 2017.


  
  %%% references for pixels chapter


%\bibitem{pixel_performance} CMS Tracker Group. ``The Performance plots for Phase 1 Pixel Detector 2017'' \url{https://twiki.cern.ch/twiki/bin/view/CMSPublic/PixelOfflinePlotsAugust2017#Alignment_of_the_forward_pixels}, last accessed on 01.05.2018
%\bibitem{unl_sop} UNL Silicon pixel group ``Pixel Phase-I activities at University of Nebraska-Lincoln (UNL)''\url{https://twiki.cern.ch/twiki/bin/view/CMS/UNLPixelPhaseI}, last accessed on 01.05.2018.
%\bibitem{aerotech} Aerotech (n.d.). ``AGS15000 Series'', retrieved from \url{https://www.aerotech.com/product-catalog/gantry-system/ags15000.aspx?p=\%2fproduct-catalog\%2fgantry-system.aspx\%3f}
%\bibitem{pr_algorithm} C. Fangmeier. ``Fiducial recognition'', 2016. Github repository,\url{https://github.com/cfangmeier/Small/blob/master/JupyterNotebooks/Fiducial_Recognition.ipynb}
%\bibitem{sop_103} F. Meier and J. Monroy `` SOP 103 - Module assembly: Gluing of HDI to BBM'', retrieved from \url{https://twiki.cern.ch/twiki/pub/CMS/UNLPixelPhaseI/UNL-PxPhI-SOP-103-v0.pdf}, last accessed on 05.30.2018.
%\bibitem{gluing_frank} F. Meier `` Forward Pixel Module Glueing at UNL'' YouTube, Sep. 15, 2014, retrieved from \url{https://www.youtube.com/watch?v=ofdntTIwKY4}.  
%\bibitem{jmonroy_channel} J. Monroy (n.d). Home [YouTube Channel]. Retrieved from \url{https://www.youtube.com/channel/UCi7S7vhYpieLOy2KJ0SS0eg}.
%\bibitem{sop_105} F. Meier and J. Monroy `` SOP 105 - Module assembly: Encapsulation of wirebonds'', retrieved from \url{https://twiki.cern.ch/twiki/pub/CMS/UNLPixelPhaseI/UNL-PxPhI-SOP-103-v0.pdf}, last accessed on 05.30.2018.
%\bibitem{fpix_module_testing_guide} J. Antonelli. ``FPix Module Testing Reference Guide'', Sep. 2015, retrieved from \url{https://cms-docdb.cern.ch/cgi-bin/DocDB/RetrieveFile?docid=12690&filename=fpix-module-testing-reference-guide.pdf&version=5} 
  
\end{thebibliography}
  
%\bibliographystyle{abbrv}
%use following if all content of bibtex file should be shown
%\nocite{*}
%\bibliography{../sections/tf}
%% Index go here (if you have one)
\end{document}

\endinput
%%
%% End of file `skeleton.tex'.
